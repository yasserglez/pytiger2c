\documentclass{article}

\usepackage[spanish]{babel}
\usepackage[utf8]{inputenc}
\usepackage{fullpage}
\usepackage{fancyvrb}
\usepackage{color}

\input{common/pygments}

\title{
    \LARGE{PyTiger2C} \\
    \Large{Anotaciones sobre el código generado}
}

\author{
    Yasser González Fernández \\
    \small{yglez@uh.cu}
    \and
    Ariel Hernández Amador \\
    \small{gnuaha7@uh.cu}
}

\date{}

\begin{document}

\maketitle

\thispagestyle{empty}

\newpage

\setcounter{page}{1}

\section{Introducción}

Nuestro proyecto se propone desarrollar una implementación de un compilador para
el lenguaje de programación Tiger que genere código \emph{C}. Posteriormente, el
código \emph{C} generado se compilará con un compilador de \emph{C} para generar
un ejecutable para una plataforma específica. El código \emph{C} generado por
nuestro compilador será conforme al \emph{standard} \emph{ISO/IEC 9899:1999},
comúnmente conocido como \emph{C99}, lo cual garantiza que pueda ser procesado
por cualquier compilador de \emph{C} conforme a dicho \emph{standard}.

Este documento brinda una descripción general de la estructura y las
características del código \emph{C} que generará nuestro compilador.

\section{Identificadores}

Un identificador en \textit{Tiger} es una secuencia de letras, dígitos y
\textit{underscores}, comenzando siempre por una letra. Según la descripción
anterior, un identificador en \textit{Tiger} es completamente válido en el
lenguaje \textit{C}.

Nuestro código \textit{C} generado tratará de asignarle a un identificador
válido de \textit{Tiger} otro con el mismo nombre en \textit{C}, siempre que
este no coincida con una palabra reservada del propio lenguaje \textit{C} o con
otro identificador definido anteriormente. En caso de que el identificador no
sea válido, se le añadirán \textit{underscores} al final hasta lograr un
identificador válido.

\section{Comentarios}

Los comentarios en \textit{Tiger} puede aparecer entre cualquier par
\textit{tokens} del lenguaje, enmarcándose entre \texttt{/*} y \texttt{*/}. El
código \textit{C} generado por nuestro compilador no reflejará los comentarios
del programa \textit{Tiger} original.

\section{Declaraciones de tipos}

\subsection{Tipos predefinidos}

El lenguaje \textit{Tiger} cuenta con dos tipos predefinidos: \texttt{int} para
los números enteros y \texttt{string} para las cadenas de caracteres. 

El código \textit{C} generado por nuestro compilador creará una variable de
tipo \texttt{int64\_t} para cada variable de tipo \texttt{int} en el programa
\textit{Tiger} de origen.

Por otra parte, a las variables de tipo \texttt{string} de un programa
\textit{Tiger} se les asociará una estructura llamada \texttt{string} cuya
definición se muestra a continuación.

\begin{quote}
\begin{Verbatim}[commandchars=@\[\]]
@PYay[struct] string {
    @PYaJ[char] @PYbe[*]data;
    @PYaJ[size_t] length;
}
\end{Verbatim}

\end{quote}

El campo \texttt{data} corresponde a la secuencia de caracteres de la cadena y
el campo \texttt{length} almacena a la longitud de la misma.

\subsection{\emph{Records}}

En \textit{Tiger} los \textit{record} son definidos por una lista de sus campos
encerrados entre llaves. Cada elemento de esta lista corresponde a la
descripción de un campo y tiene la forma \verb|field_name: field_type| donde
\texttt{field\_type} es un identificador definido con anterioridad o de el
propio tipo del \emph{record}.

El código \textit{C} generado por nuestro compilador contendrá una estructura
para cada \textit{record} definido en el programa \textit{Tiger} de origen. 
Campo de la estructura corresponderá con uno equivalente en el \textit{record}
definido en \emph{Tiger}. En caso de que algún campo tenga un nombre no válido
en \emph{C}, se seguirá la misma estrategia de renombramiento que en el caso de
los identificadores.

El siguiente ejemplo muestra el código \textit{C} generado para la definición
del \textit{record} \texttt{people} en \emph{Tiger}.

\begin{quote}
\begin{Verbatim}[commandchars=@\[\]]
@PYay[type] people { name: @PYaJ[string], age: @PYaJ[int] }
\end{Verbatim}

\end{quote}

\begin{quote}
\begin{Verbatim}[commandchars=@\[\]]
@PYay[struct] people {
    @PYay[struct] string @PYbe[*]name;
    @PYaJ[int64_t] age;
}
\end{Verbatim}

\end{quote}

El lenguaje \textit{Tiger} permite la declaración de \textit{records}
mutuamente recursivos, que son definidos en función de ellos mismos. Esta
característica de \textit{Tiger} no trae ninguna complicación adicional al
código \textit{C} equivalente, pues las estructuras de \textit{C} también
pueden contener campos del mismo tipo de la estructura que se está definiendo.

El siguiente ejemplo muestra el código \textit{C} generado equivalente a la
definición de \textit{record} correspondiente a un árbol binario
(\texttt{bianry\_tree}).

\begin{quote}
\begin{Verbatim}[commandchars=@\[\]]
@PYay[type] binary_tree { value: @PYaJ[int], left: binary_tree, right: binary_tree }
\end{Verbatim}

\end{quote}

\begin{quote}
\begin{Verbatim}[commandchars=@\[\]]
@PYay[struct] binary_tree {
    @PYaJ[int64_t] value;
    @PYay[struct] binary_tree @PYbe[*]left;
    @PYay[struct] binary_tree @PYbe[*]right;
}
\end{Verbatim}

\end{quote}

\subsection{\emph{Arrays}}

En el lenguaje \textit{Tiger} es posible declarar \textit{arrays} de cualquier
tipo previamente declarado utilizando la sintaxis \texttt{array of}
\textit{type\_name}. Nuestro código \textit{C} creará una estructura
semejante a la usada para el manejo del tipo básico \texttt{string} que
almacenará en \texttt{data} un puntero al primer elemento de la secuencia de
datos y en \texttt{length} la cantidad de elementos de el mismo.

El siguiente ejemplo ilustra la creación de un \textit{array} en \textit{Tiger}
y el código \textit{C} equivalente generado.

\begin{quote}
\begin{Verbatim}[commandchars=@\[\]]
@PYay[type] integers = @PYay[array] @PYay[of] @PYaJ[int]
\end{Verbatim}

\end{quote}

\begin{quote}
\begin{Verbatim}[commandchars=@\[\]]
@PYay[struct] integers {
    @PYaJ[uint64_t] @PYbe[*]data;
    @PYaJ[size_t] length;
}
\end{Verbatim}

\end{quote}

\section{Funciones}

En \textit{Tiger} existen dos tipos de funciones, las que no tienen valor de
retorno, a las cuales se denomina \textit{procedimientos} y las que tienen un
valor y tipo de retorno que se denominan propiamente \textit{funciones}. En
nuestro código \textit{C} generado para ambas se sigue la misma idea, con la
diferencia de que los procedimientos son generados como funciones
\texttt{void}, por lo que nos referiremos a los procedimientos como otra
función cualquiera.

Tanto las funciones como los procedimientos de \textit{Tiger} definen su propio
ámbito (\textit{scope}) y a su vez tienen acceso a los identificadores y tipos
definidos en el ámbito en que fue definido (su \textit{scope} padre).

En nuestro código \textit{C} generado, para cada ámbito se declarará una
estructura de \textit{C}, que se nombrará \\ \texttt{función\_scope}, con
campos para todas las variables declaradas en este y una referencia a la
estructura correspondiente al ámbito donde fue definida la función. En caso de
que existan conflictos en el nombre de la estructura se seguirá la misma
estrategia de renombramiento antes expuesta.

Los fragmentos de código \textit{Tiger} que no se encuentren en el
cuerpo de una función, se tratarán de modo especial, generando su
código \textit{C} equivalente como cuerpo de la función \texttt{main}. En este
caso tambien se creará una estructura que defina el ámbito correspondiente,
llamada \texttt{main\_scope}, con la única diferencia que no tendrá referencia
al ámbito padre.

Una función de \textit{Tiger} tendrá como equivalente una función de \textit{C}
de igual nombre, cuyo valor de retorno será del tipo correspondiente al de la
función de \textit{Tiger} original y \texttt{void} en el caso de los
procedimientos. Esta función recibirá como primer parámetro la estructura
correspondiente al ámbito padre y a continuación los parámetros equivalentes a
los que recibe la función de \textit{Tiger} original. En caso de que existan
conflictos con el nombre de la función se seguirá la misma estrategia de
renombramiento antes expuesta.

Los siguientes ejemplos muestran algunas situaciones de las antes expuestas y
el código \textit{C} correspondiente.

\begin{itemize}

    \item Código que no esté en el cuerpo de ninguna función.
    
    \begin{quote}
    \input{codegen/tig-main_f}
    \end{quote}

    Código \textit{C} equivalente.

    \begin{quote}
    \input{codegen/c-main_f}
    \end{quote}

    \item Declaración de una función simple.
    
    \begin{quote}
    \input{codegen/tig-function}
    \end{quote}

    Código \textit{C} equivalente.

    \begin{quote}
    \begin{Verbatim}[commandchars=@\[\]]
@PYay[struct] main_scope
{
    @PYaJ[int64_t] c;
};

@PYay[struct] f_scope
{
    @PYay[struct] main_scope@PYbe[*] parent;
    @PYaJ[int64_t] a;
    @PYaJ[int64_t] b;
};

@PYaJ[int64_t] f(@PYay[struct] main_scope@PYbe[*] parent, @PYaJ[int64_t] a, @PYaJ[int64_t] b);

@PYaJ[int64_t] @PYaL[f](@PYay[struct] main_scope@PYbe[*] parent, @PYaJ[int64_t] a, @PYaJ[int64_t] b)
{
    @PYay[struct] f_scope@PYbe[*] scope;
    @PYaJ[int64_t] local_var;

    @PYaE[/* Reservar memoria para la estructura scope */]

    scope@PYbe[-]@PYbe[>]parent @PYbe[=] parent;
    scope@PYbe[-]@PYbe[>]a @PYbe[=] a;
    scope@PYbe[-]@PYbe[>]b @PYbe[=] b;
    local_var@PYbe[=] scope@PYbe[-]@PYbe[>]a @PYbe[+] scope@PYbe[-]@PYbe[>]b;

    @PYaE[/* Liberar memoria de la estructura scope */]

    @PYay[return] local_var;
}

@PYaJ[int] @PYaL[main]()
{
    @PYay[struct] main_scope@PYbe[*] scope;
    @PYaJ[int64_t] local_var;

    @PYaE[/* Reservar memoria para la estructura scope */]

    scope@PYbe[-]@PYbe[>]c @PYbe[=]  @PYag[0];
    local_var @PYbe[=] f(scope, @PYag[5], @PYag[10]);
    scope@PYbe[-]@PYbe[>]c @PYbe[=] local_var;

    @PYaE[/* Liberar memoria de la estructura scope */]

}
\end{Verbatim}

    \end{quote} 

\end{itemize}

En \textit{Tiger} se permiten funciones recursivas y funciones anidadas
(definidas en el cuerpo de otra función), pero ninguna de estas características
constituyen un impedimento para la estrategia de generación de código antes
expuesta. El siguiente ejemplo muestra una función anidad y el código
\textit{C} equivalente. 

    \begin{quote}
    \begin{Verbatim}[commandchars=@\[\]]
@PYay[let] 
    @PYay[var] c := @PYag[0]
    @PYay[function] f(v: @PYaJ[int]):@PYaJ[int] = 
        @PYay[let] 
            @PYay[function] g(h: @PYaJ[int]):@PYaJ[int] = v + h
        @PYay[in]
            g(@PYag[5])
        @PYay[end]
@PYay[in]
    c := f(@PYag[10])
@PYay[end]
\end{Verbatim}

    \end{quote}

    Código \textit{C} equivalente.

    \begin{quote}
    \begin{Verbatim}[commandchars=@\[\]]
@PYay[struct] main_scope
{
    @PYaJ[int64_t] c;
};

@PYay[struct] f_scope
{
    @PYay[struct] main_scope@PYbe[*] parent;
    @PYaJ[int64_t] v;
};

@PYay[struct] g_scope
{
    @PYay[struct] f_scope@PYbe[*] parent;
    @PYaJ[int64_t] h;
};

@PYaJ[int64_t] f(@PYay[struct] main_scope@PYbe[*] parent, @PYaJ[int64_t] v);
@PYaJ[int64_t] g(@PYay[struct] f_scope@PYbe[*] parent, @PYaJ[int64_t] h);

@PYaJ[int64_t] @PYaL[f](@PYay[struct] main_scope@PYbe[*] parent, @PYaJ[int64_t] v)
{
    @PYay[struct] f_scope@PYbe[*] scope;
    @PYaJ[int64_t] local_var;
    @PYaJ[int64_t] local_var1;

    @PYaE[/* Reservar memoria para la estructura scope */]

    scope@PYbe[-]@PYbe[>]parent @PYbe[=] parent;
    scope@PYbe[-]@PYbe[>]v @PYbe[=] v;

    local_var @PYbe[=] g(scope, @PYag[5]);
    local_var1@PYbe[=](local_var);

    @PYaE[/* Liberar memoria de la estructura scope */]

    @PYay[return] local_var1;
}

@PYaJ[int64_t] @PYaL[g](@PYay[struct] f_scope@PYbe[*] parent, @PYaJ[int64_t] h)
{
    @PYay[struct] g_scope@PYbe[*] scope;
    @PYaJ[int64_t] local_var;

    @PYaE[/* Reservar memoria para la estructura scope */]

    scope@PYbe[-]@PYbe[>]parent @PYbe[=] parent;
    scope@PYbe[-]@PYbe[>]h @PYbe[=] h;
    local_var @PYbe[=] scope@PYbe[-]@PYbe[>]parent@PYbe[-]@PYbe[>]v @PYbe[+] scope@PYbe[-]@PYbe[>]h;

    @PYaE[/* Liberar memoria de la estructura scope */]

    @PYay[return] local_var;
}

@PYaJ[int] @PYaL[main]()
{
    @PYay[struct] main_scope@PYbe[*] scope;
    @PYaJ[int64_t] local_var;

    @PYaE[/* Reservar memoria para la estructura scope */]

    scope@PYbe[-]@PYbe[>]c @PYbe[=] @PYag[0];
    local_var @PYbe[=] f(scope, @PYag[10]);
    scope@PYbe[-]@PYbe[>]c @PYbe[=] local_var;

    @PYaE[/* Liberar memoria de la estructura scope */]

}
\end{Verbatim}

    \end{quote}

\section{Estructura general del archivo \emph{C}}

El archivo de un programa \textit{C} generado como equivalente de un programa en
\textit{Tiger} tendrá la siguiente estructura en su código.

\begin{enumerate}
    \item \texttt{\#include}s: En esta sección se incluyen las cabeceras
    correspondientes a librerías de \textit{C} que serán usadas en el programa
    generado.
    \item Declaraciones de Tipos
        \begin{enumerate}
            \item Tipos de la  \textit{Biblioteca Standard}: En esta parte del
            código se encontrarán las declaraciones de los tipos
            \texttt{string} e \texttt{int} además de cualquier tipo que sea
            necesitado por la librería standard.
      		\item Tipos del Programa: En esta parte del código se encontrarán
      		las declaraciones de los tipos definidos en el programa de origen.
  		\end{enumerate}
  	\item Declaración de los \textit{Scopes}. En esta parte del código se
  	encontrarán las declaraciones de los Scopes unidos a cada función del
  	programa \textit{Tiger} de origen.
  	\item Prototipos de las funciones: Al colocar los prototipos en esta parte
  	del código se garantiza que las funciones serán accesible para las que los
  	necesiten sin importar el orden.
        \begin{enumerate}
            \item Prototipos de las funciones de la \textit{Biblioteca
            Standard}: En esta parte del código se encuentran la descripción de
            todas las funciones de la librería standard.
            \item Prototipos de las funciones del programa: En esta parte del
            código se muestran la descripción de las funciones que son
            definidas en el programa \textit{Tiger} original.
        \end{enumerate}
    \item Cuerpo de las funciones
        \begin{itemize}
            \item Cuerpo de las funciones de la biblioteca standard: En esta
            parte del código se encontrarán las implementaciones de las
            funciones de la biblioteca standard de \textit{Tiger}.
            \item Cuerpo de las funciones del programa: En esta parte del código
            estará el código \textit{C} equivalente a cada función definida en
            el programa \textit{Tiger} de origen.
        \end{itemize}
    \item Función \textit{main}: Esta función recibe un trato especial y en su
    cuerpo se encontrará el código \textit{C} equivalente las instrucciones que
    no se encuentran dentro de una declaración de función o de tipo. 
\end{enumerate}
\end{document}
