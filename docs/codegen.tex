\documentclass{article}

\usepackage[spanish]{babel}
\usepackage[utf8]{inputenc}
\usepackage{fullpage}
\usepackage{fancyvrb}
\usepackage{color}

\input{common/pygments}

\title{
    \LARGE{PyTiger2C} \\
    \Large{Anotaciones sobre el código generado}
}

\author{
    Yasser González Fernández \\
    \small{yglez@uh.cu}
    \and
    Ariel Hernández Amador \\
    \small{gnuaha7@uh.cu}
}

\date{}

\begin{document}

\maketitle

\thispagestyle{empty}

\newpage

\setcounter{page}{1}

\section{Introducción}

% Dar una introducción sobre el código que se va a generar y 
% la idea general que se va a seguir.

\section{Identificadores}

% Escribir que el nombre de un identificador en Tiger se va a 
% traducir a un identificador en C si este no coincide con una
% palabra reservada o con un identificador utilizado anteriormente.
% Si un identificador especificado en Tiger no es válido (por las
% razones anteriores) se añadirán underscores al final del nombre
% hasta que sea válido.

\section{Comentarios}

% Escribir que los comentarios hechos en el programa Tiger
% no se van a escribir en el programa C.

\section{Declaraciones de tipos}

% Comentarios sobre las declaraciones de tipos en general.

\subsection{Tipos predefinidos}

% El tipo int de Tiger se va a traducir a un tipo int64_t de C. 

% El tipo string de Tiger se va a traducir a una estructura que
% guarda la cadena como un char *  y la longitud de la cadena.

\begin{quote}
\begin{Verbatim}[commandchars=@\[\]]
@PYay[struct] string {
    @PYaJ[char] @PYbe[*]data;
    @PYaJ[size_t] length;
}
\end{Verbatim}

\end{quote}

\subsection{\emph{Records}}

% Records en Tiger equivalen a estructuras en C.

\begin{quote}
\begin{Verbatim}[commandchars=@\[\]]
@PYay[type] persona { nombre: @PYaJ[string], edad: @PYaJ[int] }
\end{Verbatim}

\end{quote}

\begin{quote}
\begin{Verbatim}[commandchars=@\[\]]
@PYay[struct] persona {
    @PYay[struct] string @PYbe[*]nombre;
    @PYaJ[uint64_t] edad;
}
\end{Verbatim}

\end{quote}

% Comentarios sobre declaración recursiva de records
% (records con campos del tipo del record que se está declarando)
% Poner ejemplo.

\subsection{\emph{Arrays}}

% Arrays en Tigers traducidos a arrays en C con la memoria
% asignada dinámicamente. Cada record se traduce a un estructura 
% en C que guarda un puntero al primer elemento y la cantidad de
% elementos del array. 

\begin{quote}
\begin{Verbatim}[commandchars=@\[\]]
@PYay[type] integers = @PYay[array] @PYay[of] @PYaJ[int]
\end{Verbatim}

\end{quote}

\begin{quote}
\begin{Verbatim}[commandchars=@\[\]]
@PYay[struct] integers {
    @PYaJ[uint64_t] @PYbe[*]data;
    @PYaJ[size_t] length;
}
\end{Verbatim}

\end{quote}

\section{Funciones}

% Una función en Tiger se traduce a una función en C que recibe su 
% scope padre y todos sus argumentos. El valor de retorno será del tipo
% correspondiente para las funciones y void apra los procedimientos.

% Para cada scope se declara una estructura funcion_scope que tiene
% una referencia al scope padre y campos para todas las variables
% declaradas en el scope. 

% Lo primero que hace el código C de una función es crear su scope.

% Poner ejemplo del código Tiger de una función y su traducción a C.

% Explicar funcionamiento de esta técnica para funciones anidadas
% y funciones recursivas.

\section{Estructura general del archivo C}

% 1. #includes.
% 2. Declaraciones de tipos.
%    2.1 Declaraciones de tipos de la biblioteca standard.
%    2.2 Declaraciones de tipos del programa.
% 3. Declaraciones de los scopes.
% 4. Prototipos de las funciones.
%    4.1 Prototipos de las funciones de la biblioteca standard.
%    4.2 Prototipos de las funciones del programa.
% 5. Cuerpo de las funciones.
%    5.1 Cuerpo de las funciones de la biblioteca standard.
%    5.2 Cuerpo de las funciones del programa.
% 6. Función main.

\end{document}