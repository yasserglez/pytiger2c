\documentclass{article}

\usepackage[spanish]{babel}
\usepackage[utf8]{inputenc}
\usepackage{fullpage}
\usepackage{fancyvrb}
\usepackage{color}

\newcommand\at{@}
\newcommand\lb{[}
\newcommand\rb{]}
\newcommand\PYbg[1]{\textcolor[rgb]{0.00,0.50,0.00}{\textbf{#1}}}
\newcommand\PYbf[1]{\textcolor[rgb]{0.73,0.40,0.53}{\textbf{#1}}}
\newcommand\PYbe[1]{\textcolor[rgb]{0.82,0.25,0.23}{\textbf{#1}}}
\newcommand\PYbd[1]{\textcolor[rgb]{0.40,0.40,0.40}{#1}}
\newcommand\PYbc[1]{\textcolor[rgb]{0.73,0.13,0.13}{#1}}
\newcommand\PYbb[1]{\textcolor[rgb]{0.00,0.50,0.00}{#1}}
\newcommand\PYba[1]{\textcolor[rgb]{0.00,0.00,0.50}{\textbf{#1}}}
\newcommand\PYaJ[1]{\textcolor[rgb]{0.69,0.00,0.25}{#1}}
\newcommand\PYaK[1]{\textcolor[rgb]{0.73,0.13,0.13}{#1}}
\newcommand\PYaH[1]{\textcolor[rgb]{0.50,0.00,0.50}{\textbf{#1}}}
\newcommand\PYaI[1]{\fcolorbox[rgb]{1.00,0.00,0.00}{1,1,1}{#1}}
\newcommand\PYaN[1]{\textcolor[rgb]{0.74,0.48,0.00}{#1}}
\newcommand\PYaO[1]{\textcolor[rgb]{0.00,0.00,1.00}{\textbf{#1}}}
\newcommand\PYaL[1]{\textcolor[rgb]{0.00,0.00,1.00}{#1}}
\newcommand\PYaM[1]{\textcolor[rgb]{0.73,0.73,0.73}{#1}}
\newcommand\PYaB[1]{\textcolor[rgb]{0.73,0.13,0.13}{#1}}
\newcommand\PYaC[1]{\textcolor[rgb]{0.67,0.13,1.00}{#1}}
\newcommand\PYaA[1]{\textcolor[rgb]{0.00,0.50,0.00}{#1}}
\newcommand\PYaF[1]{\textcolor[rgb]{0.63,0.00,0.00}{#1}}
\newcommand\PYaG[1]{\textcolor[rgb]{1.00,0.00,0.00}{#1}}
\newcommand\PYaD[1]{\textcolor[rgb]{0.00,0.50,0.00}{\textbf{#1}}}
\newcommand\PYaE[1]{\textcolor[rgb]{0.25,0.50,0.50}{\textit{#1}}}
\newcommand\PYaZ[1]{\textcolor[rgb]{0.00,0.50,0.00}{\textbf{#1}}}
\newcommand\PYaX[1]{\textcolor[rgb]{0.00,0.50,0.00}{#1}}
\newcommand\PYaY[1]{\textcolor[rgb]{0.73,0.13,0.13}{#1}}
\newcommand\PYaR[1]{\textcolor[rgb]{0.40,0.40,0.40}{#1}}
\newcommand\PYaS[1]{\textcolor[rgb]{0.10,0.09,0.49}{#1}}
\newcommand\PYaP[1]{\textcolor[rgb]{0.00,0.00,0.50}{\textbf{#1}}}
\newcommand\PYaQ[1]{\textcolor[rgb]{0.49,0.56,0.16}{#1}}
\newcommand\PYaV[1]{\textcolor[rgb]{0.00,0.00,1.00}{\textbf{#1}}}
\newcommand\PYaW[1]{\textcolor[rgb]{0.73,0.13,0.13}{#1}}
\newcommand\PYaT[1]{\textcolor[rgb]{0.40,0.40,0.40}{#1}}
\newcommand\PYaU[1]{\textcolor[rgb]{0.25,0.50,0.50}{\textit{#1}}}
\newcommand\PYaj[1]{\textcolor[rgb]{0.00,0.50,0.00}{#1}}
\newcommand\PYak[1]{\textcolor[rgb]{0.73,0.40,0.53}{#1}}
\newcommand\PYah[1]{\textcolor[rgb]{0.63,0.63,0.00}{#1}}
\newcommand\PYai[1]{\textcolor[rgb]{0.10,0.09,0.49}{#1}}
\newcommand\PYan[1]{\textcolor[rgb]{0.40,0.40,0.40}{#1}}
\newcommand\PYao[1]{\textcolor[rgb]{0.73,0.40,0.13}{\textbf{#1}}}
\newcommand\PYal[1]{\textcolor[rgb]{0.25,0.50,0.50}{\textit{#1}}}
\newcommand\PYam[1]{\textbf{#1}}
\newcommand\PYab[1]{\textit{#1}}
\newcommand\PYac[1]{\textcolor[rgb]{0.73,0.13,0.13}{#1}}
\newcommand\PYaa[1]{\textcolor[rgb]{0.50,0.50,0.50}{#1}}
\newcommand\PYaf[1]{\textcolor[rgb]{0.25,0.50,0.50}{\textit{#1}}}
\newcommand\PYag[1]{\textcolor[rgb]{0.40,0.40,0.40}{#1}}
\newcommand\PYad[1]{\textcolor[rgb]{0.00,0.25,0.82}{#1}}
\newcommand\PYae[1]{\textcolor[rgb]{0.40,0.40,0.40}{#1}}
\newcommand\PYaz[1]{\textcolor[rgb]{0.00,0.63,0.00}{#1}}
\newcommand\PYax[1]{\textcolor[rgb]{0.60,0.60,0.60}{\textbf{#1}}}
\newcommand\PYay[1]{\textcolor[rgb]{0.00,0.50,0.00}{\textbf{#1}}}
\newcommand\PYar[1]{\textcolor[rgb]{0.10,0.09,0.49}{#1}}
\newcommand\PYas[1]{\textcolor[rgb]{0.73,0.13,0.13}{\textit{#1}}}
\newcommand\PYap[1]{\textcolor[rgb]{0.00,0.50,0.00}{\textbf{#1}}}
\newcommand\PYaq[1]{\textcolor[rgb]{0.53,0.00,0.00}{#1}}
\newcommand\PYav[1]{\textcolor[rgb]{0.67,0.13,1.00}{\textbf{#1}}}
\newcommand\PYaw[1]{\textcolor[rgb]{0.40,0.40,0.40}{#1}}
\newcommand\PYat[1]{\textcolor[rgb]{0.10,0.09,0.49}{#1}}
\newcommand\PYau[1]{\textcolor[rgb]{0.10,0.09,0.49}{#1}}

\title{
    \LARGE{PyTiger2C} \\
    \Large{Anotaciones sobre el código generado}
}

\author{
    Yasser González Fernández \\
    \small{yglez@uh.cu}
    \and
    Ariel Hernández Amador \\
    \small{gnuaha7@uh.cu}
}

\date{}

\begin{document}

\maketitle

\thispagestyle{empty}

\newpage

\setcounter{page}{1}

\section{Introducción}

% Dar una introducción sobre el código que se va a generar y 
% la idea general que se va a seguir.

\section{Identificadores}

En \textit{Tiger} los identificadores son una secuencia de letras, dígitos y
\textit{underscores (``\_'')}, comenzando siempre por una letra. Según la
descripción anterior, un identificador de \textit{Tiger} es completamente
válido en el lenguaje \textit{C}.

Nuestro código \textit{C} generado tratará de asignarle a un identificador
válido de \textit{Tiger} otro con el mismo nombre en \textit{C} siempre que
este no coincida con una palabra reservada del propio lenguaje \textit{C} o con
otro identificador definido anteriormente. En caso de que el identificador no
sea válido ,se le añadirán \textit{underscores (``\_'')} al final de este hasta
que lograr un nombre de identificador válido.

\section{Comentarios}

Un comentario en \textit{Tiger} puede aparecer entre cualquier par
\textit{tokens} del lenguaje, enmarcándose entre /* al inicio y */ al final.
Nuestro código \textit{C} generado no reflejará los comentarios del programa
\textit{Tiger} original.

\section{Declaraciones de tipos}

% Comentarios sobre las declaraciones de tipos en general.

\subsection{Tipos predefinidos}

El lenguaje \textit{Tiger} cuenta con los tipos \texttt{int} (números enteros)
y \texttt{string} (cadena de caracteres) como únicos tipos predefinidos.

Nuestro código \textit{C} generado va crear una variable de tipo
\texttt{int64\_t} para cada variable de tipo \texttt{int} en el programa
\textit{Tiger} de origen.

Por otra parte a las variables de tipo \texttt{string} de un programa
\textit{Tiger} se les asociará una estructura llamada \texttt{string} cuya
definición se representa a continuación, donde el campo \texttt{data}
corresponde a la secuencia de caracteres de esta y el campo \texttt{length}
corresponde a la longitud de la misma.

\begin{quote}
\begin{Verbatim}[commandchars=@\[\]]
@PYay[struct] string {
    @PYaJ[char] @PYbe[*]data;
    @PYaJ[size_t] length;
}
\end{Verbatim}

\end{quote}

\subsection{\emph{Records}}

En \textit{Tiger} los tipos \textit{record} son definidos por una lista de sus
campos encerrados entre llaves (\{\}), donde cada elemento de esta lista
corresponde a la descripción de un campo y tiene la forma 
\textit{nombre\_del\_campo}:\textit{Id\_tipo} donde \textit{Id\_tipo} es un
identificador definido con anterioridad.

Nuestro código \textit{C} generado creará una estructura (\texttt{struct}) para
cada \textit{record} definido en el programa \textit{Tiger} de origen, donde
cada campo de la estructura corresponderá con uno equivalente en el
\textit{record}. En caso de que algún campo tenga un nombre no válido, se
seguirá la misma estrategia de renombramiento que en el caso de los
identificadores.

El siguiente ejemplo muestra el código \textit{C} generado equivalente para la
definición de \textit{record} correspondiente a una \texttt{persona} dada. 

\begin{quote}
\begin{Verbatim}[commandchars=@\[\]]
@PYay[type] persona { nombre: @PYaJ[string], edad: @PYaJ[int] }
\end{Verbatim}

\end{quote}

\begin{quote}
\begin{Verbatim}[commandchars=@\[\]]
@PYay[struct] persona {
    @PYay[struct] string @PYbe[*]nombre;
    @PYaJ[int64_t] edad;
}
\end{Verbatim}

\end{quote}

El lenguaje \textit{Tiger} permite la declaración de \textit{records}
mutuamente recursivos, que son definidos en función de ellos mismos. Esta
característica de \textit{Tiger} no trae ninguna complicación adicional al
código \textit{C} equivalente, pues las estructuras de \textit{C} también
pueden ser declarados en función de elloe mismos. 

El siguiente ejemplo muestra el código \textit{C} generado equivalente a la
definición de \textit{record} correspondiente a un árbol binario
(\texttt{arbol\_b}) dada.

\begin{quote}
\begin{Verbatim}[commandchars=@\[\]]
@PYay[type] arbol_b { valor: @PYaJ[int], arbol_b, hijo_der: arbol_b }
\end{Verbatim}

\end{quote}

\begin{quote}
\begin{Verbatim}[commandchars=@\[\]]
@PYay[struct] arbol_b {
    @PYaJ[int64_t] valor;
    @PYay[struct] arbol_b @PYbe[*]hijo_izq;
    @PYay[struct] arbol_b @PYbe[*]hijo_der;
}
\end{Verbatim}

\end{quote}

\subsection{\emph{Arrays}}

En \textit{Tiger} se pueden declara \textit{arrays} de cualquier tipo
previamente declarado mediante la sintaxis \texttt{array of} \textit{Id\_tipo}.
Nuestro código \textit{C} generado creará una estructura semejante a la usada
para el manejo del tipo básico \texttt{string} que almacenará en \texttt{data}
un puntero al primer elemento de la secuencia de datos y en \texttt{length} la
cantidad de elementos de el mismo.

El siguiente ejemplo ilustra como se crea un \textit{array} en \textit{Tiger} y
el código \textit{C} generado equivalente.

\begin{quote}
\begin{Verbatim}[commandchars=@\[\]]
@PYay[type] integers = @PYay[array] @PYay[of] @PYaJ[int]
\end{Verbatim}

\end{quote}

\begin{quote}
\begin{Verbatim}[commandchars=@\[\]]
@PYay[struct] integers {
    @PYaJ[int64_t] @PYbe[*]data;
    @PYaJ[size_t] length;
}
\end{Verbatim}

\end{quote}

\section{Funciones}

En \textit{Tiger} existen dos tipos de funciones, las que no tienen valor de
retorno, a las cuales se denomina \textit{procedimientos} y las que tienen un
valor y tipo de retorno que se denominan propiamente \textit{funciones}. En
nuestro código \textit{C} generado para ambas se sigue la misma idea, con la
diferencia de que los procedimientos son generados como funciones
\texttt{void}, por lo que nos referiremos a los procedimientos como otra
función cualquiera.

Tanto las funciones como los procedimientos de \textit{Tiger} definen su propio
ámbito (\textit{scope}) y a su vez tienen acceso a los identificadores y tipos
definidos en el ámbito en que fue definido (su \textit{scope} padre).

En nuestro código \textit{C} generado, para cada ámbito se declarará una
estructura de \textit{C}, que se nombrará \\ \texttt{función\_scope}, con
campos para todas las variables declaradas en este y una referencia a la
estructura correspondiente al ámbito donde fue definida la función. En caso de
que existan conflictos en el nombre de la estructura se seguirá la misma
estrategia de renombramiento antes expuesta.

Los fragmentos de código \textit{Tiger} que no se encuentren en el
cuerpo de una función, se tratarán de modo especial, generando su
código \textit{C} equivalente como cuerpo de la función \texttt{main}. En este
caso tambien se creará una estructura que defina el ámbito correspondiente,
llamada \texttt{main\_scope}, con la única diferencia que no tendrá referencia
al ámbito padre.

Una función de \textit{Tiger} tendrá como equivalente una función de \textit{C}
de igual nombre, cuyo valor de retorno será del tipo correspondiente al de la
función de \textit{Tiger} original y \texttt{void} en el caso de los
procedimientos. Esta función recibirá como primer parámetro la estructura
correspondiente al ámbito padre y a continuación los parámetros equivalentes a
los que recibe la función de \textit{Tiger} original. En caso de que existan
conflictos con el nombre de la función se seguirá la misma estrategia de
renombramiento antes expuesta.

% Poner ejemplo del código Tiger de una función y su traducción a C.

% Explicar funcionamiento de esta técnica para funciones anidadas
% y funciones recursivas.

\section{Estructura general del archivo C}

Un archivo \textit{C} generado como equivalente a un programa en \textit{Tiger}
seguirá la siguiente estructura en su código.

\begin{enumerate}
    \item \texttt{\#include}s: En esta sección se incluyen las cabeceras
    correspondientes a librerías de \textit{C} que serán usadas en el programa
    generado.
    \item Declaraciones de Tipos
        \begin{enumerate}
            \item Tipos de la  \textit{Biblioteca Standard}: En esta parte del
            código se encontrarán las declaraciones de los tipos
            \texttt{string} e \texttt{int} además de cualquier tipo que sea
            necesitado por la librería standard.
      		\item Tipos del Programa: En esta parte del código se encontrarán
      		las declaraciones de los tipos definidos en el programa de origen.
  		\end{enumerate}
  	\item Declaración de los \textit{Scopes}. En esta parte del código se
  	encontrarán las declaraciones de los Scopes unidos a cada función del
  	programa \textit{Tiger} de origen.
  	\item Prototipos de las funciones: Al colocar los prototipos en esta parte
  	del código se garantiza que las funciones serán accesible para las que los
  	necesiten sin importar el orden.
        \begin{enumerate}
            \item Prototipos de las funciones de la \textit{Biblioteca
            Standard}: En esta parte del código se encuentran la descripción de
            todas las funciones de la librería standard.
            \item Prototipos de las funciones del programa: En esta parte del
            código se muestran la descripción de las funciones que son
            definidas en el programa \textit{Tiger} original.
        \end{enumerate}
    \item Cuerpo de las funciones
        \begin{itemize}
            \item Cuerpo de las funciones de la biblioteca standard: En esta
            parte del código se encontrarán las implementaciones de las
            funciones de la biblioteca standard de \textit{Tiger}.
            \item Cuerpo de las funciones del programa: En esta parte del código
            estará el código \textit{C} equivalente a cada función definida en
            el programa \textit{Tiger} de origen.
        \end{itemize}
    \item Función \textit{main}: Esta función recibe un trato especial y en su
    cuerpo se encontrará el código \textit{C} equivalente las instrucciones que
    no se encuentran dentro de una declaración de función o de tipo. 
\end{enumerate}

\end{document}