\documentclass{article}

\usepackage[spanish]{babel}
\usepackage[utf8]{inputenc}
\usepackage{fullpage}
\usepackage{fancyvrb}
\usepackage{color}

\newcommand\at{@}
\newcommand\lb{[}
\newcommand\rb{]}
\newcommand\PYbg[1]{\textcolor[rgb]{0.00,0.50,0.00}{\textbf{#1}}}
\newcommand\PYbf[1]{\textcolor[rgb]{0.73,0.40,0.53}{\textbf{#1}}}
\newcommand\PYbe[1]{\textcolor[rgb]{0.82,0.25,0.23}{\textbf{#1}}}
\newcommand\PYbd[1]{\textcolor[rgb]{0.40,0.40,0.40}{#1}}
\newcommand\PYbc[1]{\textcolor[rgb]{0.73,0.13,0.13}{#1}}
\newcommand\PYbb[1]{\textcolor[rgb]{0.00,0.50,0.00}{#1}}
\newcommand\PYba[1]{\textcolor[rgb]{0.00,0.00,0.50}{\textbf{#1}}}
\newcommand\PYaJ[1]{\textcolor[rgb]{0.69,0.00,0.25}{#1}}
\newcommand\PYaK[1]{\textcolor[rgb]{0.73,0.13,0.13}{#1}}
\newcommand\PYaH[1]{\textcolor[rgb]{0.50,0.00,0.50}{\textbf{#1}}}
\newcommand\PYaI[1]{\fcolorbox[rgb]{1.00,0.00,0.00}{1,1,1}{#1}}
\newcommand\PYaN[1]{\textcolor[rgb]{0.74,0.48,0.00}{#1}}
\newcommand\PYaO[1]{\textcolor[rgb]{0.00,0.00,1.00}{\textbf{#1}}}
\newcommand\PYaL[1]{\textcolor[rgb]{0.00,0.00,1.00}{#1}}
\newcommand\PYaM[1]{\textcolor[rgb]{0.73,0.73,0.73}{#1}}
\newcommand\PYaB[1]{\textcolor[rgb]{0.73,0.13,0.13}{#1}}
\newcommand\PYaC[1]{\textcolor[rgb]{0.67,0.13,1.00}{#1}}
\newcommand\PYaA[1]{\textcolor[rgb]{0.00,0.50,0.00}{#1}}
\newcommand\PYaF[1]{\textcolor[rgb]{0.63,0.00,0.00}{#1}}
\newcommand\PYaG[1]{\textcolor[rgb]{1.00,0.00,0.00}{#1}}
\newcommand\PYaD[1]{\textcolor[rgb]{0.00,0.50,0.00}{\textbf{#1}}}
\newcommand\PYaE[1]{\textcolor[rgb]{0.25,0.50,0.50}{\textit{#1}}}
\newcommand\PYaZ[1]{\textcolor[rgb]{0.00,0.50,0.00}{\textbf{#1}}}
\newcommand\PYaX[1]{\textcolor[rgb]{0.00,0.50,0.00}{#1}}
\newcommand\PYaY[1]{\textcolor[rgb]{0.73,0.13,0.13}{#1}}
\newcommand\PYaR[1]{\textcolor[rgb]{0.40,0.40,0.40}{#1}}
\newcommand\PYaS[1]{\textcolor[rgb]{0.10,0.09,0.49}{#1}}
\newcommand\PYaP[1]{\textcolor[rgb]{0.00,0.00,0.50}{\textbf{#1}}}
\newcommand\PYaQ[1]{\textcolor[rgb]{0.49,0.56,0.16}{#1}}
\newcommand\PYaV[1]{\textcolor[rgb]{0.00,0.00,1.00}{\textbf{#1}}}
\newcommand\PYaW[1]{\textcolor[rgb]{0.73,0.13,0.13}{#1}}
\newcommand\PYaT[1]{\textcolor[rgb]{0.40,0.40,0.40}{#1}}
\newcommand\PYaU[1]{\textcolor[rgb]{0.25,0.50,0.50}{\textit{#1}}}
\newcommand\PYaj[1]{\textcolor[rgb]{0.00,0.50,0.00}{#1}}
\newcommand\PYak[1]{\textcolor[rgb]{0.73,0.40,0.53}{#1}}
\newcommand\PYah[1]{\textcolor[rgb]{0.63,0.63,0.00}{#1}}
\newcommand\PYai[1]{\textcolor[rgb]{0.10,0.09,0.49}{#1}}
\newcommand\PYan[1]{\textcolor[rgb]{0.40,0.40,0.40}{#1}}
\newcommand\PYao[1]{\textcolor[rgb]{0.73,0.40,0.13}{\textbf{#1}}}
\newcommand\PYal[1]{\textcolor[rgb]{0.25,0.50,0.50}{\textit{#1}}}
\newcommand\PYam[1]{\textbf{#1}}
\newcommand\PYab[1]{\textit{#1}}
\newcommand\PYac[1]{\textcolor[rgb]{0.73,0.13,0.13}{#1}}
\newcommand\PYaa[1]{\textcolor[rgb]{0.50,0.50,0.50}{#1}}
\newcommand\PYaf[1]{\textcolor[rgb]{0.25,0.50,0.50}{\textit{#1}}}
\newcommand\PYag[1]{\textcolor[rgb]{0.40,0.40,0.40}{#1}}
\newcommand\PYad[1]{\textcolor[rgb]{0.00,0.25,0.82}{#1}}
\newcommand\PYae[1]{\textcolor[rgb]{0.40,0.40,0.40}{#1}}
\newcommand\PYaz[1]{\textcolor[rgb]{0.00,0.63,0.00}{#1}}
\newcommand\PYax[1]{\textcolor[rgb]{0.60,0.60,0.60}{\textbf{#1}}}
\newcommand\PYay[1]{\textcolor[rgb]{0.00,0.50,0.00}{\textbf{#1}}}
\newcommand\PYar[1]{\textcolor[rgb]{0.10,0.09,0.49}{#1}}
\newcommand\PYas[1]{\textcolor[rgb]{0.73,0.13,0.13}{\textit{#1}}}
\newcommand\PYap[1]{\textcolor[rgb]{0.00,0.50,0.00}{\textbf{#1}}}
\newcommand\PYaq[1]{\textcolor[rgb]{0.53,0.00,0.00}{#1}}
\newcommand\PYav[1]{\textcolor[rgb]{0.67,0.13,1.00}{\textbf{#1}}}
\newcommand\PYaw[1]{\textcolor[rgb]{0.40,0.40,0.40}{#1}}
\newcommand\PYat[1]{\textcolor[rgb]{0.10,0.09,0.49}{#1}}
\newcommand\PYau[1]{\textcolor[rgb]{0.10,0.09,0.49}{#1}}

\title{
    \LARGE{PyTiger2C} \\
    \Large{Anotaciones sobre el código generado}
}

\author{
    Yasser González Fernández \\
    \small{yglez@uh.cu}
    \and
    Ariel Hernández Amador \\
    \small{gnuaha7@uh.cu}
}

\date{}

\begin{document}

\maketitle

\thispagestyle{empty}

\newpage

\setcounter{page}{1}

\section{Introducción}

Nuestro proyecto se propone desarrollar una implementación de un compilador para
el lenguaje de programación Tiger que genere código \emph{C}. Posteriormente, el
código \emph{C} generado se compilará con un compilador de \emph{C} para generar
un ejecutable para la plataforma específica. El código \emph{C} generado por
nuestro compilador será conforme al \emph{standard} \emph{ISO/IEC 9899:1999},
comúnmente conocido como \emph{C99}, lo cual garantiza que pueda ser procesado
por cualquier compilador de \emph{C} que implemente dicho \emph{standard}.

Este documento brinda una descripción general de la estructura y las
características del código \emph{C} que generará nuestro compilador.

\section{Identificadores}

Un identificador en \textit{Tiger} es una secuencia de letras, dígitos y
\textit{underscores}, comenzando siempre por una letra. Según la descripción
anterior, un identificador en \textit{Tiger} es completamente válido en el
lenguaje \textit{C}.

En el código \textit{C} generado se tratará de asignar a un identificador
válido de \textit{Tiger} otro con el mismo nombre en \textit{C}, siempre que
este no coincida con una palabra reservada del propio lenguaje \textit{C} o con
otro identificador definido anteriormente. En caso de que el identificador no
sea válido, se le añadirán \textit{underscores} al final hasta lograr un 
identificador válido.

\section{Comentarios}

Los comentarios en \textit{Tiger} puede aparecer entre cualquier par
\textit{tokens} del lenguaje, enmarcándose entre \texttt{/*} y \texttt{*/}. El
código \textit{C} generado por nuestro compilador no reflejará los comentarios
del programa \textit{Tiger} original.

\section{Declaraciones de tipos}

\subsection{Tipos predefinidos}

El lenguaje \textit{Tiger} cuenta con dos tipos básicos predefinidos:
\texttt{int} para números enteros y \texttt{string} para las cadenas de
caracteres.

El código \textit{C} generado por nuestro compilador creará una variable de
tipo \texttt{int64\_t} para cada variable de tipo \texttt{int} en el programa
\textit{Tiger} de origen.

Por otra parte, a las variables de tipo \texttt{string} de un programa
\textit{Tiger} se les asociará una estructura llamada \texttt{string} cuya
definición se muestra a continuación.

\begin{quote}
\begin{Verbatim}[commandchars=@\[\]]
@PYay[struct] tiger_string {
    @PYaJ[char] @PYbe[*]data;
    @PYaJ[size_t] length;
}
\end{Verbatim}

\end{quote}

El campo \texttt{data} corresponde a la secuencia de caracteres de la cadena y
el campo \texttt{length} almacena a la longitud de la misma.

\subsection{\emph{Records}}

En \textit{Tiger} los \textit{record} son definidos por una lista de sus campos
encerrados entre llaves. Cada elemento de esta lista corresponde a la
descripción de un campo y tiene la forma \verb|field_name: field_type| donde
\texttt{field\_type} es un identificador definido con anterioridad o de el
propio tipo del \emph{record}.

El código \textit{C} generado por nuestro compilador contendrá una estructura
para cada \textit{record} definido en el programa \textit{Tiger} de origen. 
Cada campo de la estructura corresponderá con uno equivalente en el
\textit{record} definido en \emph{Tiger}. En caso de que algún campo tenga un
nombre no válido en \emph{C}, se seguirá la misma estrategia de renombramiento
que en el caso de los identificadores.

El siguiente ejemplo muestra el código \textit{C} generado para la definición
del \textit{record} \texttt{people} en \emph{Tiger} y su representación en el
lenguaje \emph{C}.

\begin{quote}
\begin{Verbatim}[commandchars=@\[\]]
@PYay[type] people { name: @PYaJ[string], age: @PYaJ[int] }
\end{Verbatim}

\end{quote}

\begin{quote}
\begin{Verbatim}[commandchars=@\[\]]
@PYay[struct] people {
    @PYay[struct] string @PYbe[*]name;
    @PYaJ[int64_t] age;
}
\end{Verbatim}

\end{quote}

El lenguaje \textit{Tiger} permite la declaración de \textit{records}
que tengan campos del propio tipo del \emph{record}, es decir, son definidos en
función de ellos mismos. Esta característica de \textit{Tiger} no conlleva
ninguna complicación adicional al código \textit{C} equivalente, pues las
estructuras de \textit{C} también pueden contener campos del mismo tipo de la
estructura que se está definiendo.

El siguiente ejemplo muestra el código \textit{C} generado para la definición
de \textit{record} correspondiente a un árbol binario.

\begin{quote}
\begin{Verbatim}[commandchars=@\[\]]
@PYay[type] binary_tree { value: @PYaJ[int], left: binary_tree, right: binary_tree }
\end{Verbatim}

\end{quote}

\begin{quote}
\begin{Verbatim}[commandchars=@\[\]]
@PYay[struct] binary_tree {
    @PYaJ[int] value;
    @PYay[struct] binary_tree @PYbe[*]left;
    @PYay[struct] binary_tree @PYbe[*]right;
}
\end{Verbatim}

\end{quote}

\subsection{\emph{Arrays}}

En el lenguaje \textit{Tiger} es posible declarar \textit{arrays} de cualquier
tipo previamente definido utilizando la sintaxis \texttt{array of}
\texttt{type\_name}. Nuestro código \textit{C} contendrá una estructura
semejante a la usada para el manejo del tipo básico \texttt{string} que
almacenará en \texttt{data} un puntero al primer elemento de la secuencia de
datos del \emph{array} y en \texttt{length} la cantidad de elementos de el
mismo.

El siguiente ejemplo ilustra la creación de un \textit{array} en \textit{Tiger}
y el código \textit{C} equivalente generado.

\begin{quote}
\begin{Verbatim}[commandchars=@\[\]]
@PYay[type] integers = @PYay[array] @PYay[of] @PYaJ[int]
\end{Verbatim}

\end{quote}

\begin{quote}
\begin{Verbatim}[commandchars=@\[\]]
@PYay[struct] integers {
    @PYaJ[int64_t] @PYbe[*]data;
    @PYaJ[size_t] length;
}
\end{Verbatim}

\end{quote}

\section{Funciones}

En \textit{Tiger} existen dos tipos de funciones, las que no tienen valor de
retorno, a las cuales se denomina \textit{procedimientos} y las que tienen un
valor y tipo de retorno que se denominan propiamente \textit{funciones}. En
nuestro código \textit{C} generado para ambas se sigue la misma idea, con la
diferencia de que los procedimientos son generados como funciones
\texttt{void}, por lo que nos referiremos a los procedimientos como otra
función cualquiera.

Tanto las funciones como los procedimientos de \textit{Tiger} definen su propio
ámbito o \textit{scope} y a su vez tienen acceso a los identificadores y tipos
definidos en el ámbito en que fue definido o su \textit{scope} padre.

En nuestro código \textit{C} generado, para cada función se declarará una
estructura \texttt{function\_name\_scope}, que tendrá campos para todas las
variables declaradas en este y una referencia a la estructura correspondiente
al ámbito donde fue definida la función. En caso de que existan conflictos en
el nombre de la estructura se seguirá la misma estrategia de renombramiento
antes expuesta.

Los fragmentos de código \textit{Tiger} que no se encuentren en el
cuerpo de una función, se tratarán de modo especial, generando su
código \textit{C} equivalente como cuerpo de la función \texttt{main}. En este
caso también se creará una estructura que defina el ámbito correspondiente,
llamada \texttt{main\_scope}, con la única diferencia que no tendrá referencia
al ámbito padre.

Una función de \textit{Tiger} tendrá como equivalente una función de \textit{C}
de igual nombre, cuyo valor de retorno será del tipo correspondiente al de la
función de \textit{Tiger} original y \texttt{void} en el caso de los
procedimientos. Esta función recibirá como primer parámetro la estructura
correspondiente al ámbito padre y a continuación los parámetros equivalentes a
los que recibe la función de \textit{Tiger} original. En caso de que existan
conflictos con el nombre de la función se seguirá la misma estrategia de
renombramiento antes expuesta.

A continuación se muestran algunas definiciones de funciones \emph{Tiger} y el
código \emph{C} generado para estas.

\begin{itemize}
  \item Código que no se encuentre en el cuerpo de ninguna función.
    
    \begin{quote}
    \begin{Verbatim}[commandchars=@\[\]]
@PYay[let]
    @PYay[var] a := @PYag[5]
    @PYay[var] b := @PYag[10]
    @PYay[var] c := @PYag[0]
@PYay[in]
    c := a + b
@PYay[end]
\end{Verbatim}

    \end{quote}
    
    \begin{quote}
    \begin{Verbatim}[commandchars=@\[\]]
@PYay[struct] scope1 {
    @PYaJ[int] a;
    @PYaJ[int] b;
    @PYaJ[int] c;
};

@PYaJ[int] @PYaL[main]()
{
    @PYay[struct] scope1@PYbe[*] scope;
    
    @PYaE[/* Reservar memoria para la estructura scope. */]

    scope@PYbe[-]@PYbe[>]a @PYbe[=] @PYag[5];
    scope@PYbe[-]@PYbe[>]b @PYbe[=] @PYag[10];
    scope@PYbe[-]@PYbe[>]c @PYbe[=] @PYag[0];
    scope@PYbe[-]@PYbe[>]c @PYbe[=] scope@PYbe[-]@PYbe[>]a @PYbe[+] scope@PYbe[-]@PYbe[>]b;
}
\end{Verbatim}

    \end{quote}
    
  \item Declaración de una función simple.
    
    \begin{quote}
    \begin{Verbatim}[commandchars=@\[\]]
@PYay[let] 
    @PYay[var] c := @PYag[0]
    @PYay[function] f(a: @PYaJ[int], b:@PYaJ[int]): @PYaJ[int] =  a + b
@PYay[in]
    c := f(@PYag[5], @PYag[10])
@PYay[end]
\end{Verbatim}

    \end{quote}
    
    \begin{quote}
    \begin{Verbatim}[commandchars=@\[\]]
@PYay[struct] scope1 {
    @PYaJ[int] c;
};

@PYay[struct] scope2 {
    @PYay[struct] scope1 @PYbe[*]parent;
    @PYaJ[int] a;
    @PYaJ[int] b;
};

@PYaJ[int] @PYaL[f](@PYay[struct] scope1 @PYbe[*]parent, @PYaJ[int] a, @PYaJ[int] b)
{
    @PYay[struct] scope2 @PYbe[*]scope;
    @PYaJ[int] local_var1;

    @PYaE[/* Reservar memoria para la estructura scope. */]

    scope@PYbe[-]@PYbe[>]parent @PYbe[=] parent;
    scope@PYbe[-]@PYbe[>]a @PYbe[=] a;
    scope@PYbe[-]@PYbe[>]b @PYbe[=] b;
    local_var1 @PYbe[=] scope@PYbe[-]@PYbe[>]a @PYbe[+] scope@PYbe[-]@PYbe[>]b;

    @PYay[return] local_var1;
}

@PYaJ[int] @PYaL[main]()
{
    @PYay[struct] scope1 @PYbe[*]scope;

    @PYaE[/* Reservar memoria para la estructura scope. */]

    scope@PYbe[-]@PYbe[>]c @PYbe[=]  @PYag[0];
    scope@PYbe[-]@PYbe[>]c @PYbe[=] f(scope, @PYag[5], @PYag[10]);
}
\end{Verbatim}

    \end{quote} 
\end{itemize}

En \textit{Tiger} se permiten funciones recursivas y funciones definidas en el
cuerpo de otra función o también llamadas funciones anidadas. Ninguna de estas
características constituyen un impedimento para la estrategia de generación de
código antes expuesta. El siguiente ejemplo muestra una función anidada y el
código \textit{C} equivalente.

\begin{quote}
\begin{Verbatim}[commandchars=@\[\]]
@PYay[let] 
    @PYay[var] c := @PYag[0]
    @PYay[function] f(v: @PYaJ[int]): @PYaJ[int] = 
        @PYay[let] 
            @PYay[function] g(h: @PYaJ[int]): @PYaJ[int] = v + h
        @PYay[in]
            g(@PYag[5])
        @PYay[end]
@PYay[in]
    c := f(@PYag[10])
@PYay[end]
\end{Verbatim}

\end{quote}

\begin{quote}
\begin{Verbatim}[commandchars=@\[\]]
@PYay[struct] main_scope {
    @PYaJ[int64_t] c;
};

@PYay[struct] f_scope {
    @PYay[struct] main_scope @PYbe[*]parent;
    @PYaJ[int64_t] v;
};

@PYay[struct] g_scope {
    @PYay[struct] f_scope @PYbe[*]parent;
    @PYaJ[int64_t] h;
};

@PYaJ[int64_t] @PYaL[f](@PYay[struct] main_scope @PYbe[*]parent, @PYaJ[int64_t] v)
{
    @PYay[struct] f_scope @PYbe[*]scope;
    @PYaJ[int64_t] local_var1;

    @PYaE[/* Reservar memoria para la estructura scope. */]

    scope@PYbe[-]@PYbe[>]parent @PYbe[=] parent;
    scope@PYbe[-]@PYbe[>]v @PYbe[=] v;
    local_var1 @PYbe[=] g(scope, @PYag[5]);

    @PYaE[/* Liberar memoria de la estructura scope. */]

    @PYay[return] local_var1;
}

@PYaJ[int64_t] @PYaL[g](@PYay[struct] f_scope @PYbe[*]parent, @PYaJ[int64_t] h)
{
    @PYay[struct] g_scope @PYbe[*]scope;
    @PYaJ[int64_t] local_var1;

    @PYaE[/* Reservar memoria para la estructura scope. */]

    scope@PYbe[-]@PYbe[>]parent @PYbe[=] parent;
    scope@PYbe[-]@PYbe[>]h @PYbe[=] h;
    local_var1 @PYbe[=] scope@PYbe[-]@PYbe[>]parent@PYbe[-]@PYbe[>]v @PYbe[+] scope@PYbe[-]@PYbe[>]h;

    @PYaE[/* Liberar memoria de la estructura scope. */]

    @PYay[return] local_var1;
}

@PYaJ[int] @PYaL[main]()
{
    @PYay[struct] main_scope @PYbe[*]scope;

    @PYaE[/* Reservar memoria para la estructura scope. */]

    scope@PYbe[-]@PYbe[>]c @PYbe[=] @PYag[0];
    scope@PYbe[-]@PYbe[>]c @PYbe[=] f(scope, @PYag[10]);

    @PYaE[/* Liberar memoria de la estructura scope. */]
}
\end{Verbatim}

\end{quote}

\section{Estructura general del archivo \emph{C}}

El código \emph{C} generado por nuestro compilador, para cualquier programa
\emph{Tiger}, estará contenido en un único archivo \emph{C} y no deberá tener
ninguna dependencia externa además de los \emph{headers} de la biblioteca
\emph{standard} de \emph{C}. Este comportamiento hace posible que sólo sea
necesario copiar el archivo \emph{C} generado hacia otra máquina y compilar
este código sin la necesidad de instalar \emph{headers} o bibliotecas
relacionadas con nuestro compilador en esta otra máquina. 

Para cumplir con este requerimiento, el archivo \emph{C} generado debe incluir
las definiciones de las funciones de la biblioteca \emph{standard} de
\emph{Tiger} utilizadas en el programa.

De manera general, el archivo de un programa \textit{C} generado como
equivalente de un programa en \textit{Tiger} tendrá la siguiente estructura en
su código.

\begin{enumerate}
    \item \textbf{Inclusión de los \emph{headers} correspondientes a las
    funciones de la biblioteca \emph{standard} de \emph{C} utilizados en el
    programa.}

    \item \textbf{Declaraciones de tipos.}
    \begin{enumerate}
        \item \textbf{Tipos de la biblioteca \emph{standard} de \emph{Tiger}.}
        En esta parte del código se encontrarán las declaraciones de los tipos
        \texttt{int} y \texttt{string} además de cualquier tipo que sea
        necesario para la implementación de las funciones de la biblioteca
        \emph{standard} de \emph{Tiger}.
  		    
  		\item \textbf{Tipos definidos en el programa.} En esta parte del código se
  		encontrarán las declaraciones equivalentes de los tipos definidos en el
  		programa \emph{Tiger} de origen.
    \end{enumerate}
    
  	\item \textbf{Declaración de los \textit{scopes}.} En esta parte del código
  	se encontrarán las declaraciones de los \emph{scopes} asociados a cada función
  	del programa \textit{Tiger} de origen y a la función \texttt{main} de
  	\emph{C}.
  
  	\item \textbf{Prototipos de las funciones.} Al colocar los prototipos en
  	esta parte del código se garantiza que las funciones serán accesible para
  	las que los necesiten sin importar el orden en que estas sean definidas más
  	adelante.
    \begin{enumerate}
        \item \textbf{Prototipos de las funciones de la biblioteca
        \emph{standard} de \emph{Tiger}}.
        \item \textbf{Prototipos de las funciones definidas en el programa
        \emph{Tiger}.}
    \end{enumerate}
    
    \item \textbf{Cuerpo de las funciones.}
    \begin{itemize}
        \item \textbf{Cuerpo de las funciones de la biblioteca \emph{standard}.}
        En esta parte del código se encontrarán las implementaciones de las
        funciones de la biblioteca \emph{standard} de \textit{Tiger} que se
        utilicen en el programa \emph{Tiger} en cuestión.
        \item \textbf{Cuerpo de las funciones definidas en el programa
        \emph{Tiger}}. En esta parte del código estará el código \textit{C}
        equivalente a cada función definida en el programa \textit{Tiger} de
        origen.
    \end{itemize}
    
    \item \textbf{Función \textit{main}}. Esta función recibe un trato especial
    y en su cuerpo se encontrará el código \textit{C} equivalente las
    instrucciones que no se encuentran dentro de una declaración de función o
    de tipo.
\end{enumerate}
\end{document}
