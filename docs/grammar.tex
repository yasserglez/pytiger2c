\documentclass{article}

\usepackage[spanish]{babel}
\usepackage[utf8]{inputenc}
\usepackage{fullpage}

\title{
	\LARGE{PyTiger2C} \\
	\Large{Gramática LR del lenguaje Tiger utilizada por PyTiger2C}
}

\author{
  	Yasser González Fernández \\
  	\small{yglez@uh.cu}
  	\and
  	Ariel Hernández Amador \\
  	\small{gnuaha7@uh.cu}
}

\date{}

\begin{document}

\maketitle

\thispagestyle{empty}

\newpage

\setcounter{page}{1}

A continuación se enumeran las reglas de la gramática libre del contexto
utilizada por PyTiger2C para reconocer las estructuras del lenguaje Tiger.

La gramática presenta algunas ambigüedades que se reportan en forma de 
conflictos \emph{SHIFT-REDUCE} al hacer el análisis LR. En algunos casos, 
es posible hacer modificaciones a la gramática para evitar estos conflictos 
pero decidimos no hacer dichas modificaciones y en su lugar utilizar reglas 
de precedencia para evitar perder claridad en las estructuras del lenguaje 
Tiger que reconocen cada una de las producciones.

Las reglas de precedencia indican que acción se debe tomar ante un conflicto 
\emph{SHIFT-REDUCE}, indicando si se debe introducir en la pila el 
\emph{token} de la cadena de entrada o reducir la producción con la que 
se produce el conflicto.

\begin{enumerate}

\item \verb|program : expr|

\item \verb|expr : NIL|

\item \verb|expr : INTLIT|

\item \verb|expr : STRLIT|

\item \verb|expr : lvalue|

\item \verb|expr : ID LBRACKET expr RBRACKET OF expr|

\item \verb|expr : ID LBRACE field_list RBRACE|

\item \verb|expr : MINUS expr|

\item \verb|expr : expr PLUS expr|

\item \verb|expr : expr MINUS expr|

\item \verb|expr : expr TIMES expr|

\item \verb|expr : expr DIVIDE expr|

\item \verb|expr : expr EQ expr|

\item \verb|expr : expr NE expr|

\item \verb|expr : expr LT expr|

\item \verb|expr : expr LE expr|

\item \verb|expr : expr GT expr|

\item \verb|expr : expr GE expr|

\item \verb|expr : expr AND expr|

\item \verb|expr : expr OR expr|

\item \verb|expr : LPAREN expr_seq RPAREN|

\item \verb|expr : lvalue ASSIGN expr|

\item \verb|expr : ID LPAREN expr_list RPAREN|

\item \verb|expr : IF expr THEN expr|

\item \verb|expr : IF expr THEN expr ELSE expr|

\item \verb|expr : WHILE expr DO expr|

\item \verb|expr : FOR ID ASSIGN expr TO expr DO expr|

\item \verb|expr : BREAK|

\item \verb|expr : LET dec_group IN expr_seq END|

\item \verb|lvalue : ID|

\item \verb|lvalue : lvalue PERIOD ID|

\item \verb|lvalue : ID LBRACKET expr RBRACKET|

\item \verb|lvalue : lvalue LBRACKET expr RBRACKET|

\item \verb|expr_seq :|

\item \verb|expr_seq : expr_seq SEMICOLON expr|

\item \verb|expr_seq : expr|

\item \verb|dec_group :|

\item \verb|dec_group : dec_group dec|

\item \verb|field_list :|

\item \verb|field_list : field_assign|

\item \verb|field_list : field_list COMMA field_assign|

\item \verb|field_assign : ID EQ expr|

\item \verb|expr_list :|

\item \verb|expr_list : expr_list COMMA expr|

\item \verb|expr_list : expr|

\item \verb|dec : type_dec_group|

\item \verb|dec : var_dec|

\item \verb|dec : func_dec_group|

\item \verb|func_dec_group : func_dec|

\item \verb|func_dec_group : func_dec_group func_dec|

\item \verb|type_dec_group : type_dec|

\item \verb|type_dec_group : type_dec_group type_dec|

\item \verb|type_dec : TYPE ID EQ type|

\item \verb|type : ID|

\item \verb|type : LBRACE field_types RBRACE|

\item \verb|type : ARRAY OF ID|

\item \verb|field_types :|

\item \verb|field_types : field_type|

\item \verb|field_types : field_types COMMA field_type|

\item \verb|field_type : ID COLON ID|

\item \verb|var_dec : VAR ID ASSIGN expr|

\item \verb|var_dec : VAR ID COLON ID ASSIGN expr|

\item \verb|func_dec : FUNCTION ID LPAREN field_types RPAREN EQ expr|

\item \verb|func_dec : FUNCTION ID LPAREN field_types RPAREN COLON ID EQ expr|

\end{enumerate}

\end{document}
