\documentclass{article}

\usepackage[spanish]{babel}
\usepackage[utf8]{inputenc}
\usepackage{fullpage}
\usepackage{fancyvrb}
\usepackage{color}

\newcommand\at{@}
\newcommand\lb{[}
\newcommand\rb{]}
\newcommand\PYbg[1]{\textcolor[rgb]{0.00,0.50,0.00}{\textbf{#1}}}
\newcommand\PYbf[1]{\textcolor[rgb]{0.73,0.40,0.53}{\textbf{#1}}}
\newcommand\PYbe[1]{\textcolor[rgb]{0.82,0.25,0.23}{\textbf{#1}}}
\newcommand\PYbd[1]{\textcolor[rgb]{0.40,0.40,0.40}{#1}}
\newcommand\PYbc[1]{\textcolor[rgb]{0.73,0.13,0.13}{#1}}
\newcommand\PYbb[1]{\textcolor[rgb]{0.00,0.50,0.00}{#1}}
\newcommand\PYba[1]{\textcolor[rgb]{0.00,0.00,0.50}{\textbf{#1}}}
\newcommand\PYaJ[1]{\textcolor[rgb]{0.69,0.00,0.25}{#1}}
\newcommand\PYaK[1]{\textcolor[rgb]{0.73,0.13,0.13}{#1}}
\newcommand\PYaH[1]{\textcolor[rgb]{0.50,0.00,0.50}{\textbf{#1}}}
\newcommand\PYaI[1]{\fcolorbox[rgb]{1.00,0.00,0.00}{1,1,1}{#1}}
\newcommand\PYaN[1]{\textcolor[rgb]{0.74,0.48,0.00}{#1}}
\newcommand\PYaO[1]{\textcolor[rgb]{0.00,0.00,1.00}{\textbf{#1}}}
\newcommand\PYaL[1]{\textcolor[rgb]{0.00,0.00,1.00}{#1}}
\newcommand\PYaM[1]{\textcolor[rgb]{0.73,0.73,0.73}{#1}}
\newcommand\PYaB[1]{\textcolor[rgb]{0.73,0.13,0.13}{#1}}
\newcommand\PYaC[1]{\textcolor[rgb]{0.67,0.13,1.00}{#1}}
\newcommand\PYaA[1]{\textcolor[rgb]{0.00,0.50,0.00}{#1}}
\newcommand\PYaF[1]{\textcolor[rgb]{0.63,0.00,0.00}{#1}}
\newcommand\PYaG[1]{\textcolor[rgb]{1.00,0.00,0.00}{#1}}
\newcommand\PYaD[1]{\textcolor[rgb]{0.00,0.50,0.00}{\textbf{#1}}}
\newcommand\PYaE[1]{\textcolor[rgb]{0.25,0.50,0.50}{\textit{#1}}}
\newcommand\PYaZ[1]{\textcolor[rgb]{0.00,0.50,0.00}{\textbf{#1}}}
\newcommand\PYaX[1]{\textcolor[rgb]{0.00,0.50,0.00}{#1}}
\newcommand\PYaY[1]{\textcolor[rgb]{0.73,0.13,0.13}{#1}}
\newcommand\PYaR[1]{\textcolor[rgb]{0.40,0.40,0.40}{#1}}
\newcommand\PYaS[1]{\textcolor[rgb]{0.10,0.09,0.49}{#1}}
\newcommand\PYaP[1]{\textcolor[rgb]{0.00,0.00,0.50}{\textbf{#1}}}
\newcommand\PYaQ[1]{\textcolor[rgb]{0.49,0.56,0.16}{#1}}
\newcommand\PYaV[1]{\textcolor[rgb]{0.00,0.00,1.00}{\textbf{#1}}}
\newcommand\PYaW[1]{\textcolor[rgb]{0.73,0.13,0.13}{#1}}
\newcommand\PYaT[1]{\textcolor[rgb]{0.40,0.40,0.40}{#1}}
\newcommand\PYaU[1]{\textcolor[rgb]{0.25,0.50,0.50}{\textit{#1}}}
\newcommand\PYaj[1]{\textcolor[rgb]{0.00,0.50,0.00}{#1}}
\newcommand\PYak[1]{\textcolor[rgb]{0.73,0.40,0.53}{#1}}
\newcommand\PYah[1]{\textcolor[rgb]{0.63,0.63,0.00}{#1}}
\newcommand\PYai[1]{\textcolor[rgb]{0.10,0.09,0.49}{#1}}
\newcommand\PYan[1]{\textcolor[rgb]{0.40,0.40,0.40}{#1}}
\newcommand\PYao[1]{\textcolor[rgb]{0.73,0.40,0.13}{\textbf{#1}}}
\newcommand\PYal[1]{\textcolor[rgb]{0.25,0.50,0.50}{\textit{#1}}}
\newcommand\PYam[1]{\textbf{#1}}
\newcommand\PYab[1]{\textit{#1}}
\newcommand\PYac[1]{\textcolor[rgb]{0.73,0.13,0.13}{#1}}
\newcommand\PYaa[1]{\textcolor[rgb]{0.50,0.50,0.50}{#1}}
\newcommand\PYaf[1]{\textcolor[rgb]{0.25,0.50,0.50}{\textit{#1}}}
\newcommand\PYag[1]{\textcolor[rgb]{0.40,0.40,0.40}{#1}}
\newcommand\PYad[1]{\textcolor[rgb]{0.00,0.25,0.82}{#1}}
\newcommand\PYae[1]{\textcolor[rgb]{0.40,0.40,0.40}{#1}}
\newcommand\PYaz[1]{\textcolor[rgb]{0.00,0.63,0.00}{#1}}
\newcommand\PYax[1]{\textcolor[rgb]{0.60,0.60,0.60}{\textbf{#1}}}
\newcommand\PYay[1]{\textcolor[rgb]{0.00,0.50,0.00}{\textbf{#1}}}
\newcommand\PYar[1]{\textcolor[rgb]{0.10,0.09,0.49}{#1}}
\newcommand\PYas[1]{\textcolor[rgb]{0.73,0.13,0.13}{\textit{#1}}}
\newcommand\PYap[1]{\textcolor[rgb]{0.00,0.50,0.00}{\textbf{#1}}}
\newcommand\PYaq[1]{\textcolor[rgb]{0.53,0.00,0.00}{#1}}
\newcommand\PYav[1]{\textcolor[rgb]{0.67,0.13,1.00}{\textbf{#1}}}
\newcommand\PYaw[1]{\textcolor[rgb]{0.40,0.40,0.40}{#1}}
\newcommand\PYat[1]{\textcolor[rgb]{0.10,0.09,0.49}{#1}}
\newcommand\PYau[1]{\textcolor[rgb]{0.10,0.09,0.49}{#1}}

\title{
	\LARGE{PyTiger2C} \\
	\Large{Anotaciones sobre la comprobación sintáctica}
}

\author{
  	Yasser González Fernández \\
  	\small{ygonzalezfernandez@gmail.com}
  	\and
  	Ariel Hernández Amador \\
  	\small{gnuaha7@gmail.com}
}

\date{}

\begin{document}

\maketitle

\thispagestyle{empty}

\newpage

\setcounter{page}{1}

\section{Análisis lexicográfico}

Durante el análisis lexicográfico se separa el flujo de caracteres de entrada
en un conjunto de \emph{tokens}. A continuación se enumeran las expresiones
regulares utilizadas para reconocer cada \emph{token} del lenguaje Tiger.

\begin{enumerate}
  \item \verb|ARRAY : array|

  \item \verb|IF : if|

  \item \verb|THEN : then|

  \item \verb|ELSE : else|

  \item \verb|WHILE : while|

  \item \verb|FOR : for|

  \item \verb|TO : to|

  \item \verb|DO : do|

  \item \verb|LET : let|

  \item \verb|IN : in|

  \item \verb|END : end|

  \item \verb|OF : of|

  \item \verb|BREAK : break|

  \item \verb|NIL : nil|

  \item \verb|FUNCTION : function|

  \item \verb|VAR : var|

  \item \verb|TYPE : type|

  \item \verb|ID : [a-zA-Z][a-zA-Z0-9_]*|

  \item \verb|STRLIT :|
  \begin{verbatim}
  "(
      (\\[nt])
    | (\\")
    | (\\\\)
    | (\\\^[@A-Z[\]^_])
    | (\\[0-9][0-9][0-9])
    | (\\\s+\\)
    | ([^\\"])
  )*"
  \end{verbatim}

  \item \verb|INTLIT : [0-9]+|

  \item \verb|PLUS : \+|

  \item \verb|MINUS : \-|

  \item \verb|TIMES : \*|

  \item \verb|DIVIDE : /|

  \item \verb|EQ : =|

  \item \verb|NE : <>|

  \item \verb|LT : <|

  \item \verb|LE : <=|

  \item \verb|GT : >|

  \item \verb|GE : >=|

  \item \verb|AND : &|

  \item \verb$OR : \|$

  \item \verb|ASSIGN : :=|

  \item \verb|PERIOD : \.|

  \item \verb|COMMA : ,|

  \item \verb|COLON : :|

  \item \verb|SEMICOLON : ;|

  \item \verb|LPAREN : \(|

  \item \verb|RPAREN : \)|

  \item \verb|LBRACKET : \[|

  \item \verb|RBRACKET : \]|

  \item \verb|LBRACE : \{|

  \item \verb|RBRACE : \}|
\end{enumerate}

\section{Análisis sintáctico}

El análisis sintáctico reconoce estructuras del lenguaje a partir del flujo de
\emph{tokens} obtenidos del análisis lexicográfico. A continuación se enumeran
las reglas de la gramática libre del contexto utilizada por PyTiger2C para
reconocer las estructuras del lenguaje Tiger.

\begin{enumerate}

  \item \verb|program ::= expr|

  \item \verb|expr ::= NIL|

  \item \verb|expr ::= INTLIT|

  \item \verb|expr ::= STRLIT|

  \item \verb|expr ::= lvalue|

  \item \verb|expr ::= ID LBRACKET expr RBRACKET OF expr|

  \item \verb|expr ::= ID LBRACE field_list RBRACE|

  \item \verb|expr ::= MINUS expr|

  \item \verb|expr ::= expr PLUS expr|

  \item \verb|expr ::= expr MINUS expr|

  \item \verb|expr ::= expr TIMES expr|

  \item \verb|expr ::= expr DIVIDE expr|

  \item \verb|expr ::= expr EQ expr|

  \item \verb|expr ::= expr NE expr|

  \item \verb|expr ::= expr LT expr|

  \item \verb|expr ::= expr LE expr|

  \item \verb|expr ::= expr GT expr|

  \item \verb|expr ::= expr GE expr|

  \item \verb|expr ::= expr AND expr|

  \item \verb|expr ::= expr OR expr|

  \item \verb|expr ::= LPAREN expr_seq RPAREN|

  \item \verb|expr ::= lvalue ASSIGN expr|

  \item \verb|expr ::= ID LPAREN expr_list RPAREN|

  \item \verb|expr ::= IF expr THEN expr|

  \item \verb|expr ::= IF expr THEN expr ELSE expr|

  \item \verb|expr ::= WHILE expr DO expr|

  \item \verb|expr ::= FOR ID ASSIGN expr TO expr DO expr|

  \item \verb|expr ::= BREAK|

  \item \verb|expr ::= LET dec_group IN expr_seq END|

  \item \verb|lvalue ::= ID|

  \item \verb|lvalue ::= lvalue PERIOD ID|

  \item \verb|lvalue ::= ID LBRACKET expr RBRACKET|

  \item \verb|lvalue ::= lvalue LBRACKET expr RBRACKET|

  \item \verb|expr_seq ::=|

  \item \verb|expr_seq ::= expr_seq SEMICOLON expr|

  \item \verb|expr_seq ::= expr|

  \item \verb|dec_group ::=|

  \item \verb|dec_group ::= dec_group dec|

  \item \verb|field_list ::=|

  \item \verb|field_list ::= field_assign|

  \item \verb|field_list ::= field_list COMMA field_assign|

  \item \verb|field_assign ::= ID EQ expr|

  \item \verb|expr_list ::=|

  \item \verb|expr_list ::= expr_list COMMA expr|

  \item \verb|expr_list ::= expr|

  \item \verb|dec ::= type_dec_group|

  \item \verb|dec ::= var_dec|

  \item \verb|dec ::= func_dec_group|

  \item \verb|func_dec_group ::= func_dec|

  \item \verb|func_dec_group ::= func_dec_group func_dec|

  \item \verb|type_dec_group ::= type_dec|

  \item \verb|type_dec_group ::= type_dec_group type_dec|

  \item \verb|type_dec ::= TYPE ID EQ type|

  \item \verb|type ::= ID|

  \item \verb|type ::= LBRACE field_types RBRACE|

  \item \verb|type ::= ARRAY OF ID|

  \item \verb|field_types ::=|

  \item \verb|field_types ::= field_type|

  \item \verb|field_types ::= field_types COMMA field_type|

  \item \verb|field_type ::= ID COLON ID|

  \item \verb|var_dec ::= VAR ID ASSIGN expr|

  \item \verb|var_dec ::= VAR ID COLON ID ASSIGN expr|

  \item \verb|func_dec ::= FUNCTION ID LPAREN field_types RPAREN EQ expr|

  \item \verb|func_dec ::= FUNCTION ID LPAREN field_types RPAREN COLON ID EQ expr|
\end{enumerate}

\subsection{Reglas de precedencia}

La gramática anterior presenta algunas ambigüedades que se reportan en forma de
conflictos \emph{SHIFT-REDUCE} al hacer el análisis LR. En algunos casos,
es posible hacer modificaciones a la gramática para evitar estos conflictos
pero decidimos no hacer dichas modificaciones y en su lugar utilizar reglas
de precedencia para evitar perder claridad en las estructuras del lenguaje
Tiger que reconocen cada una de las producciones.

Las reglas de precedencia indican que acción se debe tomar ante un conflicto
\emph{SHIFT-REDUCE}, indicando si se debe introducir en la pila el
\emph{token} de la cadena de entrada o reducir la producción del conflicto.

Utilizando PLY, las reglas de precedencia se especifican asignando valores
de prioridad a los \emph{tokens} de la gramática mediante una lista. Un
\emph{token} en la lista tendrá mayor prioridad que todos los \emph{tokens} que
aparezcan anteriormente. Se seleccionará hacer \emph{REDUCE} a una regla o hacer
\emph{SHIFT} a un \emph{token} en favor del que tenga mayor prioridad. La
prioridad de una regla está dada por la prioridad del último \emph{token}
presente en la regla.

A continuación se muestran las reglas de precedencia utilizadas para la
gramática anterior.

\begin{quote}
\begin{Verbatim}[commandchars=@\[\]]
precedence @PYbe[=] (
    @PYaE[# The following fixes the shift/reduce conflict caused by rules ending with]
    @PYaE[# the non-terminal expr and the rules for binary operators.]
    (@PYaB[']@PYaB[nonassoc]@PYaB['], @PYaB[']@PYaB[OF]@PYaB['], @PYaB[']@PYaB[THEN]@PYaB['], @PYaB[']@PYaB[DO]@PYaB[']),
    @PYaE[# The token ELSE has higher priority to fix the dangling-else shift/reduce]
    @PYaE[# conflict. If an ELSE if found it should be shifted instead of reducing the]
    @PYaE[# if-then without the else clause.]
    (@PYaB[']@PYaB[nonassoc]@PYaB['], @PYaB[']@PYaB[ELSE]@PYaB[']),
    @PYaE[# The following fixes the shift/reduce conflict between shifting the]
    @PYaE[# @PYZlb[] token in the p_expr_array production (array literal) or reducing]
    @PYaE[# the p_lvalue_id production.]
    (@PYaB[']@PYaB[nonassoc]@PYaB['], @PYaB[']@PYaB[LVALUE_ID]@PYaB[']),
    (@PYaB[']@PYaB[nonassoc]@PYaB['], @PYaB[']@PYaB[LBRACKET]@PYaB[']),
    @PYaE[# Operator precedence.]
    (@PYaB[']@PYaB[nonassoc]@PYaB['], @PYaB[']@PYaB[ASSIGN]@PYaB[']),
    (@PYaB[']@PYaB[left]@PYaB['], @PYaB[']@PYaB[OR]@PYaB[']),
    (@PYaB[']@PYaB[left]@PYaB['], @PYaB[']@PYaB[AND]@PYaB[']),
    (@PYaB[']@PYaB[nonassoc]@PYaB['], @PYaB[']@PYaB[EQ]@PYaB['], @PYaB[']@PYaB[NE]@PYaB['], @PYaB[']@PYaB[LT]@PYaB['], @PYaB[']@PYaB[LE]@PYaB['], @PYaB[']@PYaB[GT]@PYaB['], @PYaB[']@PYaB[GE]@PYaB[']),
    (@PYaB[']@PYaB[left]@PYaB['], @PYaB[']@PYaB[PLUS]@PYaB['], @PYaB[']@PYaB[MINUS]@PYaB[']),
    (@PYaB[']@PYaB[left]@PYaB['], @PYaB[']@PYaB[TIMES]@PYaB['], @PYaB[']@PYaB[DIVIDE]@PYaB[']),
    (@PYaB[']@PYaB[right]@PYaB['], @PYaB[']@PYaB[UMINUS]@PYaB[']),
)
\end{Verbatim}

\end{quote}

Un grupo de conflictos \emph{SHIFT-REDUCE} es producido por todas las reglas
de la gramática que terminen en el no terminal \verb|expr| y las reglas
utilizadas para los operadores binarios. Se utilizan reglas de precedencia
para indicar que en estos casos se debe introducir el \emph{token} correspondiente
al operador en la pila en lugar de reducir las reglas que terminan en \verb|expr|.

Además, se produce un conflicto \emph{SHIFT-REDUCE} al reconocer las expresiones
\texttt{if-then} y \texttt{if-then-else}. Este problema es conocido en la
literatura como \emph{dangling else} y se puede solucionar introduciendo el
\emph{token} \emph{else} en la pila, indicando que este \emph{else} corresponde
a la estructura \emph{if-then} anterior.

Otro conflicto \emph{SHIFT-REDUCE} se produce con la producción que reconoce un
identificador como un \emph{lvalue} y el \emph{token} \verb|LBRACKET| en la
declaración de un literal de \emph{array}. En este caso se indica que se introduzca
el token \verb|LBRACKET| en la pila para tratar de reconocer la declaración de
un literal de \emph{array}.

El resto de las reglas de precedencia se utilizan para garantizar la precedencia
y asociatividad entre los operadores.

\end{document}
