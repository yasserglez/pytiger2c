\documentclass{article}

\usepackage[spanish]{babel}
\usepackage[utf8]{inputenc}
\usepackage{fullpage}
\usepackage{fancyvrb}
\usepackage{color}

\newcommand\at{@}
\newcommand\lb{[}
\newcommand\rb{]}
\newcommand\PYbg[1]{\textcolor[rgb]{0.00,0.50,0.00}{\textbf{#1}}}
\newcommand\PYbf[1]{\textcolor[rgb]{0.73,0.40,0.53}{\textbf{#1}}}
\newcommand\PYbe[1]{\textcolor[rgb]{0.82,0.25,0.23}{\textbf{#1}}}
\newcommand\PYbd[1]{\textcolor[rgb]{0.40,0.40,0.40}{#1}}
\newcommand\PYbc[1]{\textcolor[rgb]{0.73,0.13,0.13}{#1}}
\newcommand\PYbb[1]{\textcolor[rgb]{0.00,0.50,0.00}{#1}}
\newcommand\PYba[1]{\textcolor[rgb]{0.00,0.00,0.50}{\textbf{#1}}}
\newcommand\PYaJ[1]{\textcolor[rgb]{0.69,0.00,0.25}{#1}}
\newcommand\PYaK[1]{\textcolor[rgb]{0.73,0.13,0.13}{#1}}
\newcommand\PYaH[1]{\textcolor[rgb]{0.50,0.00,0.50}{\textbf{#1}}}
\newcommand\PYaI[1]{\fcolorbox[rgb]{1.00,0.00,0.00}{1,1,1}{#1}}
\newcommand\PYaN[1]{\textcolor[rgb]{0.74,0.48,0.00}{#1}}
\newcommand\PYaO[1]{\textcolor[rgb]{0.00,0.00,1.00}{\textbf{#1}}}
\newcommand\PYaL[1]{\textcolor[rgb]{0.00,0.00,1.00}{#1}}
\newcommand\PYaM[1]{\textcolor[rgb]{0.73,0.73,0.73}{#1}}
\newcommand\PYaB[1]{\textcolor[rgb]{0.73,0.13,0.13}{#1}}
\newcommand\PYaC[1]{\textcolor[rgb]{0.67,0.13,1.00}{#1}}
\newcommand\PYaA[1]{\textcolor[rgb]{0.00,0.50,0.00}{#1}}
\newcommand\PYaF[1]{\textcolor[rgb]{0.63,0.00,0.00}{#1}}
\newcommand\PYaG[1]{\textcolor[rgb]{1.00,0.00,0.00}{#1}}
\newcommand\PYaD[1]{\textcolor[rgb]{0.00,0.50,0.00}{\textbf{#1}}}
\newcommand\PYaE[1]{\textcolor[rgb]{0.25,0.50,0.50}{\textit{#1}}}
\newcommand\PYaZ[1]{\textcolor[rgb]{0.00,0.50,0.00}{\textbf{#1}}}
\newcommand\PYaX[1]{\textcolor[rgb]{0.00,0.50,0.00}{#1}}
\newcommand\PYaY[1]{\textcolor[rgb]{0.73,0.13,0.13}{#1}}
\newcommand\PYaR[1]{\textcolor[rgb]{0.40,0.40,0.40}{#1}}
\newcommand\PYaS[1]{\textcolor[rgb]{0.10,0.09,0.49}{#1}}
\newcommand\PYaP[1]{\textcolor[rgb]{0.00,0.00,0.50}{\textbf{#1}}}
\newcommand\PYaQ[1]{\textcolor[rgb]{0.49,0.56,0.16}{#1}}
\newcommand\PYaV[1]{\textcolor[rgb]{0.00,0.00,1.00}{\textbf{#1}}}
\newcommand\PYaW[1]{\textcolor[rgb]{0.73,0.13,0.13}{#1}}
\newcommand\PYaT[1]{\textcolor[rgb]{0.40,0.40,0.40}{#1}}
\newcommand\PYaU[1]{\textcolor[rgb]{0.25,0.50,0.50}{\textit{#1}}}
\newcommand\PYaj[1]{\textcolor[rgb]{0.00,0.50,0.00}{#1}}
\newcommand\PYak[1]{\textcolor[rgb]{0.73,0.40,0.53}{#1}}
\newcommand\PYah[1]{\textcolor[rgb]{0.63,0.63,0.00}{#1}}
\newcommand\PYai[1]{\textcolor[rgb]{0.10,0.09,0.49}{#1}}
\newcommand\PYan[1]{\textcolor[rgb]{0.40,0.40,0.40}{#1}}
\newcommand\PYao[1]{\textcolor[rgb]{0.73,0.40,0.13}{\textbf{#1}}}
\newcommand\PYal[1]{\textcolor[rgb]{0.25,0.50,0.50}{\textit{#1}}}
\newcommand\PYam[1]{\textbf{#1}}
\newcommand\PYab[1]{\textit{#1}}
\newcommand\PYac[1]{\textcolor[rgb]{0.73,0.13,0.13}{#1}}
\newcommand\PYaa[1]{\textcolor[rgb]{0.50,0.50,0.50}{#1}}
\newcommand\PYaf[1]{\textcolor[rgb]{0.25,0.50,0.50}{\textit{#1}}}
\newcommand\PYag[1]{\textcolor[rgb]{0.40,0.40,0.40}{#1}}
\newcommand\PYad[1]{\textcolor[rgb]{0.00,0.25,0.82}{#1}}
\newcommand\PYae[1]{\textcolor[rgb]{0.40,0.40,0.40}{#1}}
\newcommand\PYaz[1]{\textcolor[rgb]{0.00,0.63,0.00}{#1}}
\newcommand\PYax[1]{\textcolor[rgb]{0.60,0.60,0.60}{\textbf{#1}}}
\newcommand\PYay[1]{\textcolor[rgb]{0.00,0.50,0.00}{\textbf{#1}}}
\newcommand\PYar[1]{\textcolor[rgb]{0.10,0.09,0.49}{#1}}
\newcommand\PYas[1]{\textcolor[rgb]{0.73,0.13,0.13}{\textit{#1}}}
\newcommand\PYap[1]{\textcolor[rgb]{0.00,0.50,0.00}{\textbf{#1}}}
\newcommand\PYaq[1]{\textcolor[rgb]{0.53,0.00,0.00}{#1}}
\newcommand\PYav[1]{\textcolor[rgb]{0.67,0.13,1.00}{\textbf{#1}}}
\newcommand\PYaw[1]{\textcolor[rgb]{0.40,0.40,0.40}{#1}}
\newcommand\PYat[1]{\textcolor[rgb]{0.10,0.09,0.49}{#1}}
\newcommand\PYau[1]{\textcolor[rgb]{0.10,0.09,0.49}{#1}}

\title{
	\LARGE{PyTiger2C} \\
	\Large{Breve descripción de Python Lex-Yacc}
}

\author{
  	Yasser González Fernández \\
  	\small{yglez@uh.cu}
  	\and
  	Ariel Hernández Amador \\
  	\small{gnuaha7@uh.cu}
}

\date{}

\begin{document}

\maketitle

\thispagestyle{empty}

\newpage

\setcounter{page}{1}

\section{Introducción}

Python Lex-Yacc ó PLY es una implementación, desarrollada completamente en
Python, de las herramientas tradicionales para la construcción de compiladores
Lex y Yacc. PLY intenta seguir de manera bastante fiel la forma en que
trabajan estas herramientas, incluyendo el soporte para gramáticas LALR(1),
$\varepsilon$-producciones, resolución de ambigüedades mediante reglas de
precedencia y recuperación de errores.

Esta herramienta fue creada por David M. Beazley en el año 2001 para ser
utilizado en un curso de Introducción a Compiladores impartido por el autor en
la Universidad de Chicago. Fue utilizada por los estudiantes de este curso
para construir compiladores para un lenguaje simple con características
semejantes a Pascal. Debido a su origen relacionado con la docencia, se dedicó
en su implementación un gran esfuerzo a proveer una extensiva comprobación de
errores.

PLY no soporta funcionalidades presentes en otras herramientas para la
construcción de compiladores como construcción automática del árbol de sintaxis
abstracta ni aspectos relacionados con la comprobación semántica. PLY sólo se
ocupa del análisis lexicográfico y sintáctico.

PLY está compuesto por un paquete Python llamado \texttt{ply} que contiene
dos módulos \texttt{lex.py} y \texttt{yacc.py}. El módulo \texttt{lex.py} se
encarga del análisis lexicográfico, es decir, separa el flujo de entrada en una
secuencia de \textit{tokens} especificados mediante un conjunto de expresiones
regulares. \texttt{yacc.py} se ocupa de reconocer la sintaxis del lenguaje
especificada en la forma de una gramática libre del contexto. \texttt{yacc.py}
realiza un análisis ascendente o LR y genera las tablas utilizando LALR(1) y
opcionalmente SLR. Los dos módulos funcionan de manera conjunta, específicamente
\texttt{lex.py} presenta una función \texttt{token()} como forma de interfaz
externa que es  ejecutada repetidamente por \texttt{yacc.py} para obtener los
\textit{tokens} y  ejecutar las acciones asociadas a las producciones de la
gramática.

La principal diferencia entre \texttt{yacc.py} y la herramienta \texttt{yacc}
de UNIX es que \texttt{yacc.py} no requiere una fase de generación de código a
partir de la descripción de la gramática. Las especificaciones dadas a PLY son
programas Python válidos, esto significa que no se requieren archivos de código
fuente adicionales ni un paso especial en el proceso de construcción del
compilador para generar el código Python del compilador. PLY hace uso de los
\emph{docstrings} de Python para asociar las reglas de la gramática con sus
acciones correspondientes. Como la generación de la tabla LALR(1) puede ser un
proceso relativamente costoso, PLY guarda la tabla resultante en un archivo y en
una próxima ejecución del compilador se comprobará si la gramática ha sido
modificada para cargar la tabla generada anteriormente en lugar de volverla a
generar.

\section{Análisis lexicográfico}

\texttt{lex.py} se utiliza para separar el flujo de caracteres de la entrada en
una secuencia de \textit{tokens}. La descripción de cada uno de los
\textit{tokens} válidos se realiza mediante expresiones regulares.

Todos los \textit{tokens} válidos se deben especificar previamente en una lista
llamada \texttt{tokens}. A continuación se muestra un fragmento de código
Python en el que se definen \textit{tokens} relacionados con operaciones
aritméticas.

\begin{quote}
\begin{Verbatim}[commandchars=@\[\]]
@PYaE[# Lista que define los tokens válidos.]
tokens @PYbd[=] (@PYaB[']@PYaB[NUMBER]@PYaB['], @PYaB[']@PYaB[PLUS]@PYaB['], @PYaB[']@PYaB[MINUS]@PYaB['])

@PYaE[# Expresiones regulares para cada uno de los tokens.]
t_NUMBER @PYbd[=] @PYaB[r']@PYaB[\]@PYaB[d+]@PYaB[']
t_PLUS @PYbd[=] @PYaB[r']@PYaB[\]@PYaB[+]@PYaB[']
t_MINUS @PYbd[=] @PYaB[r']@PYaB[-]@PYaB[']
\end{Verbatim}

\end{quote}

Todos los \textit{tokens} tienen asociado un nombre determinado por los
caracteres luego del prefijo \texttt{t\_} y un valor que generalmente es el
lexema asociado al \textit{token}.

Si es necesario asociar una acción a la regla de un \textit{token} se puede
especificar el \textit{token} mediante una función. A continuación se muestra
una regla asociada al \textit{token} \texttt{NUMBER} que convierte la cadena al
entero correspondiente.

\begin{quote}
\begin{Verbatim}[commandchars=@\[\]]
@PYay[def] @PYaL[t_NUMBER](token):
    @PYaB[r']@PYaB[\]@PYaB[d+]@PYaB[']
    token@PYbd[.]value @PYbd[=] @PYaX[int](token@PYbd[.]value)
    @PYay[return] token
\end{Verbatim}

\end{quote}

Para construir el \textit{lexer} se utiliza la función \texttt{lex.lex()} que
retorna una instancia de la clase \texttt{Lexer}. Una vez que ha sido
construido, se pueden utilizar dos funciones para interactuar con el
\textit{lexer}.

\begin{itemize}
  \item \texttt{lexer.input(data)} Permite especificar el flujo de caracteres
  	de entrada.
  \item \texttt{lexer.token()} Retorna una instancia de \texttt{LexToken} que
  	representa el siguiente \textit{token} en el flujo de entrada. Si todos los
  	\textit{tokens} han sido consumidos retornará \texttt{None}.
\end{itemize}

\section{Análisis sintáctico}

\texttt{yacc.py} se utiliza para reconocer la sintaxis del lenguaje. Las
producciones de la gramática se especifican utilizando notación BNF en los
\textit{docstrings} de las funciones que corresponden a las acciones asociadas
a estas. La primera producción de la gramática especificada se tomará como el
símbolo inicial.

A continuación se muestra un ejemplo de una producción que forma parte
de un evaluador de expresiones cuya implementación completa se muestra en la
sección~\ref{sec:ejemplo}.

\begin{quote}
\begin{Verbatim}[commandchars=@\[\]]
@PYay[def] @PYaL[p_expr_plus](symbs):
    @PYaB[r']@PYaB[expr : expr PLUS term]@PYaB[']
    symbs@lb[]@PYaw[0]@rb[] @PYbd[=] symbs@lb[]@PYaw[1]@rb[] @PYbd[+] symbs@lb[]@PYaw[3]@rb[]
\end{Verbatim}

\end{quote}

La función acepta un único argumento \texttt{symbs} que es una secuencia
que contiene los valores asociados a los símbolos de la gramática presentes en
la producción. Las posiciones en la secuencia se asignan de izquierda a
derecha. Es posible asignar cualquier objeto Python como valor asociado a un
símbolo en una producción.

Para construir el \textit{parser} se utiliza la función \texttt{yacc.yacc()}
que recibe un \textit{lexer} y una cadena para reconocer y retorna una instancia
de la clase \texttt{Parser}. Una vez que ha sido construido, se puede utilizar
el método \texttt{parse()} para indicar que comience a reconocer la cadena y
ejecutar las acciones asociadas a las producciones.

\section{Ejemplo}
\label{sec:ejemplo}

A continuación se muestra un programa Python que utiliza PLY para implementar
un pequeño evaluador de expresiones aritméticas con números. Cuando este
programa se ejecuta muestra un \textit{prompt} y queda a la espera de que se
introduzca una expresión para evaluarla y mostrar su valor.

\begin{quote}
\begin{Verbatim}[commandchars=@\[\]]
@PYaE[# -*- coding: utf-8 -*-]

@PYas["""]
@PYas[Ejemplo de un evaluador de expresiones aritméticas con PLY.]
@PYas["""]

@PYay[from] @PYaV[ply] @PYay[import] lex, yacc

@PYaE[# Reglas del lexer.]

tokens @PYbd[=] (@PYaB[']@PYaB[NUMBER]@PYaB['], @PYaB[']@PYaB[PLUS]@PYaB['], @PYaB[']@PYaB[MINUS]@PYaB['], @PYaB[']@PYaB[TIMES]@PYaB['], @PYaB[']@PYaB[DIVIDE]@PYaB['], @PYaB[']@PYaB[LPAREN]@PYaB['], @PYaB[']@PYaB[RPAREN]@PYaB['])

@PYay[def] @PYaL[t_NUMBER](token):
    @PYaB[r']@PYaB[\]@PYaB[d+]@PYaB[']
    token@PYbd[.]value @PYbd[=] @PYaX[int](token@PYbd[.]value)
    @PYay[return] token

t_PLUS @PYbd[=] @PYaB[r']@PYaB[\]@PYaB[+]@PYaB[']
t_MINUS @PYbd[=] @PYaB[r']@PYaB[-]@PYaB[']
t_TIMES @PYbd[=] @PYaB[r']@PYaB[\]@PYaB[*]@PYaB[']
t_DIVIDE @PYbd[=] @PYaB[r']@PYaB[/]@PYaB[']
t_LPAREN @PYbd[=] @PYaB[r']@PYaB[\]@PYaB[(]@PYaB[']
t_RPAREN @PYbd[=] @PYaB[r']@PYaB[\]@PYaB[)]@PYaB[']
t_ignore  @PYbd[=] @PYaB[']@PYaB[ ]@PYao[\t]@PYaB[']


@PYaE[# Reglas del parser.]

@PYay[def] @PYaL[p_expr_plus](symbs):
    @PYaB[r']@PYaB[expr : expr PLUS term]@PYaB[']
    symbs@lb[]@PYaw[0]@rb[] @PYbd[=] symbs@lb[]@PYaw[1]@rb[] @PYbd[+] symbs@lb[]@PYaw[3]@rb[]
    
@PYay[def] @PYaL[p_expr_minus](symbs):
    @PYaB[r']@PYaB[expr : expr MINUS term]@PYaB[']
    symbs@lb[]@PYaw[0]@rb[] @PYbd[=] symbs@lb[]@PYaw[1]@rb[] @PYbd[-] symbs@lb[]@PYaw[3]@rb[]
    
@PYay[def] @PYaL[p_expr_term](symbs):
    @PYaB[r']@PYaB[expr : term]@PYaB[']
    symbs@lb[]@PYaw[0]@rb[] @PYbd[=] symbs@lb[]@PYaw[1]@rb[]

@PYay[def] @PYaL[p_term_times](symbs):
    @PYaB[r']@PYaB[term : term TIMES factor]@PYaB[']
    symbs@lb[]@PYaw[0]@rb[] @PYbd[=] symbs@lb[]@PYaw[1]@rb[] @PYbd[*] symbs@lb[]@PYaw[3]@rb[]
    
@PYay[def] @PYaL[p_term_divide](symbs):
    @PYaB[r']@PYaB[term : term DIVIDE factor]@PYaB[']
    symbs@lb[]@PYaw[0]@rb[] @PYbd[=] symbs@lb[]@PYaw[1]@rb[] @PYbd[/] symbs@lb[]@PYaw[3]@rb[]
    
@PYay[def] @PYaL[p_term_factor](symbs):
    @PYaB[r']@PYaB[term : factor]@PYaB[']
    symbs@lb[]@PYaw[0]@rb[] @PYbd[=] symbs@lb[]@PYaw[1]@rb[]

@PYay[def] @PYaL[p_factor_number](symbs):
    @PYaB[r']@PYaB[factor : NUMBER]@PYaB[']
    symbs@lb[]@PYaw[0]@rb[] @PYbd[=] symbs@lb[]@PYaw[1]@rb[]
    
@PYay[def] @PYaL[p_factor_paren](symbs):
    @PYaB[r']@PYaB[factor : LPAREN expr RPAREN]@PYaB[']
    symbs@lb[]@PYaw[0]@rb[] @PYbd[=] symbs@lb[]@PYaw[2]@rb[]


@PYay[def] @PYaL[main]():
    @PYas["""]
@PYas[    Función principal del script.]
@PYas[    """]
    lexer @PYbd[=] lex@PYbd[.]lex()
    parser @PYbd[=] yacc@PYbd[.]yacc()
    @PYay[while] @PYaA[True]:
        @PYay[try]:
            expression @PYbd[=] @PYaX[raw_input](@PYaB[']@PYaB[expression: ]@PYaB['])
        @PYay[except] @PYbe[EOFError]:
            @PYay[break]
        @PYay[if] expression:
            value @PYbd[=] parser@PYbd[.]parse(expression, lexer@PYbd[=]lexer)
            @PYay[print] @PYaB[']@PYaB[value: ]@PYaB['], value


@PYay[if] __name__ @PYbd[==] @PYaB[']@PYaB[__main__]@PYaB[']:
    main()
\end{Verbatim}

\end{quote}
\end{document}
