\documentclass{article}

\usepackage[spanish]{babel}
\usepackage[utf8]{inputenc}
\usepackage{fullpage}
\usepackage{graphicx}

\title{
    \LARGE{PyTiger2C} \\
    \Large{Descripción de los archivos de documentación}
}

\author{
    Yasser González Fernández \\
    \small{yglez@uh.cu}
    \and
    Ariel Hernández Amador \\
    \small{gnuaha7@uh.cu}
}

\date{}

\begin{document}

\maketitle

\thispagestyle{empty}

\newpage

\setcounter{page}{1}

A continuación se describe de manera general el contenido de los archivos de
documentación en el directorio \texttt{docs} de la distribución en código
fuente de PyTiger2C.

\begin{description}
\item[\texttt{readme.pdf}:] Contiene una descripción general de los archivos en
  el directorio \texttt{docs}. Es este documento.

\item[\texttt{gui.pdf}:] Descripción de las funcionalidades de la interfaz
  gráfica de PyTiger2C.

\item[\texttt{ply.pdf}:] Descripción de la herramienta \emph{Python Lex-Yacc}
  utilizada en la implementación del análisis lexicográfico y sintáctico.

\item[\texttt{grammar.pdf}:] Anotaciones sobre el análisis sintáctico LR.

\item[\texttt{lrtable.pdf}:] Tabla utilizada en el análisis sintáctico LR.

\item[\texttt{lrautomata.pdf}:] Autómata utilizado en el análisis sintáctico LR.

\item[\texttt{languagenodes.pdf}:] Jerarquía de las clases correspondientes
  a los nodos del árbol de sintáxis abstracta.

\item[\texttt{semantic.pdf}:] Anotaciones sobre el análisis semántico.
  Descripción general del lenguaje Tiger reconocido por PyTiger2C.

\item[\texttt{codegen.pdf}:] Anotaciones sobre la generación de código.

\item[\texttt{api.pdf}:] Documentación del API generada automáticamente a
  partir del código fuente Python.

\item[\texttt{changelog.pdf}:] Historia de los cambios realizados al proyecto
  desde el comienzo de su implementación.

\end{description}

\end{document}
